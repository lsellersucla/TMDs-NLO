\documentclass[11pt,a4paper]{article}

\usepackage{graphicx}
\usepackage{afterpage}
\usepackage{epsfig,cite}
\usepackage{amssymb}
\usepackage{amsmath}
\usepackage{dsfont}
\usepackage{multirow}
\usepackage{url,hyperref}
\usepackage{bm}

\textwidth=15.0cm \textheight=22.0cm 
\topmargin 0cm \oddsidemargin 0cm 
\setlength{\unitlength}{1mm}

\usepackage{url}
\usepackage{hyperref}

\bibliographystyle{h-elsevier3}

\usepackage{tikz}
\usetikzlibrary{shapes,arrows}

%%%%%%%%%%%%%%%%%%%%%%%%%%%%%%%%%%%%%%%%%%%%%%%%%%%%%%%%%%%%%

% Define special colors
\usepackage{color}
\definecolor{comment}{rgb}{0,0.3,0}
\definecolor{identifier}{rgb}{0.0,0,0.3}

\usepackage{listings}

\definecolor{listinggray}{gray}{0.9}
\definecolor{lbcolor}{rgb}{0.9,0.9,0.9}
\lstset{
  backgroundcolor=\color{lbcolor},
  tabsize=4,
  % rulecolor=,
  language=[GNU]C++,
  basicstyle=\scriptsize,
  upquote=true,
  aboveskip={1.5\baselineskip},
  columns=fixed,
  showstringspaces=false,
  extendedchars=false,
  breaklines=true,
  prebreak = \raisebox{0ex}[0ex][0ex]{\ensuremath{\hookleftarrow}},
  frame=single,
  numbers=left,
  showtabs=false,
  showspaces=false,
  showstringspaces=false,
  identifierstyle=\ttfamily,
  keywordstyle=\color[rgb]{0,0,1},
  commentstyle=\color[rgb]{0.026,0.112,0.095},
  stringstyle=\color[rgb]{0.627,0.126,0.941},
  numberstyle=\color[rgb]{0.205, 0.142, 0.73},
%  \lstdefinestyle{C++}{language=C++,style=numbers},
}

\lstset{
  backgroundcolor=\color{lbcolor},
  tabsize=4,
  language=C++,
  captionpos=b,
  tabsize=3,
  frame=lines,
  numbers=left,
  numberstyle=\tiny,
  numbersep=5pt,
  breaklines=true,
  showstringspaces=false,
  basicstyle=\footnotesize,
% identifierstyle=\color{magenta},
  keywordstyle=\color[rgb]{0,0,1},
  commentstyle=\color{green!40!black},
  stringstyle=\color{red}
}

%GPS suggestion
\lstset{basicstyle=\footnotesize\ttfamily,breaklines=true} 

\begin{document}

\begin{center}
{\Large \bf {\tt APFEL v2.6.1}: A PDF Evolution Library with QED corrections}
\vspace{.7cm}

Valerio~Bertone$^{1,2}$,
Stefano~Carrazza$^{2}$ and Juan~Rojo$^1$

\vspace{.3cm}
{
\it ~$^1$ Rudolf Peierls Centre for Theoretical Physics,\\
\it 1 Keble Road, University of Oxford, OX1 3NP, Oxford, UK\\
\it ~$^2$ PH Department, TH Unit, CERN, CH-1211 Geneva 23, Switzerland \\
}
\end{center}

\vspace{0.1cm}

\begin{center}
{\bf \large Abstract}
\end{center}

In this document we present the user manual for the {\tt APFEL}
library. Written in {\scshape Fortran 77}, all the functionalities can
also be accessed via the {\tt C/C++} and {\tt Python} interfaces.  For
simplicity, we will restrict ourselves to the description of the {\tt
  C/C++} interface, but the usage of the {\scshape Fortran 77} and
{\tt Python} interfaces is very similar and examples of their use are
provided in the {\tt examples} folder of the {\tt APFEL} source code.
First of all, we will discuss how to install {\tt APFEL} and how to
execute the basic example programs. After that, we will list the
various customization options that can be accessed by the user for
both the PDF evolution and the DIS structure functions modules.

\tableofcontents

\section{Installation}

The  {\tt APFEL} library is available from its {\tt HepForge} website:
\begin{center}
{\bf \url{http://apfel.hepforge.org/}~}
\end{center}
and from the {\tt GitHub} webpage:
\begin{center}
{\bf \url{https://github.com/scarrazza/apfel}~}
\end{center}
It can also be accessed directly from the {\tt git} repository.
The last development version can be downloaded by giving:
\begin{lstlisting}
  git clone https://github.com/scarrazza/apfel.git
\end{lstlisting}
For the tagged versions one can use the {\tt git} tag commands:
\begin{lstlisting}
  git tag -l
  git checkout tags/tag_name
\end{lstlisting}
to switch to any of the past releases. We strongly recommend to use
the latest stables release.

The installation of the {\tt APFEL} library can be easily done
following the standard {\tt autotools} sequence:
\begin{lstlisting}
  ./configure
  make
  make install
\end{lstlisting}
which automatically installs {\tt APFEL} in {\tt /usr/local/}. Note
that by default the {\tt APFEL} library requires an installation of
the {\tt LHAPDF} PDF library\footnote{The current release of {\tt
    APFEL} assumes that {\tt LHAPDF} version 6 has been previously
  installed as version 5 is no longer supported.}. However, an
LHAPDF-less installation is also supported by giving:
\begin{lstlisting}
  ./configure --disable-lhapdf
\end{lstlisting}
To use a different installation path, one simply needs to
use the option:
\begin{lstlisting}
  ./configure --prefix=/path/to/the/installation/folder
\end{lstlisting}
In this case, the {\tt APFEL} installation path should be included to
the environmental variable {\tt LD\_LIBRARY\_PATH}. This can be done
adding to the local {\tt .bashrc} file (or {\tt .profile} file on Mac)
the string:
\begin{lstlisting}
  export LD_LIBRARY_PATH=$LD_LIBRARY_PATH:/path/to/the/installation/folder/lib
\end{lstlisting}

Once {\tt APFEL} has been properly compiled and installed, the
configuration script {\tt apfel-config} should automatically be
present. Such script is useful to determine the compiler flags. Type:
\begin{lstlisting}
  apfel-config --help
\end{lstlisting}
in a shell to see all the possible options. Particularly useful is the
{\tt --list-funcs} flag that lists all the functions available in {\tt
  APFEL} along with a short explanation. In addition, also the shell
script {\tt apfel} is provided which starts an interactive console
session of {\tt APFEL} providing a prompt tool to use the library
without coding.

In the following we will list and illustrate all the functionalities
of {\tt APFEL} and explaining how they can be accessed by the
user. The most recent version of {\tt APFEL} provide also an
additional module to compute DIS (and SIA) structure functions in
different mass scheme. Such module is dependent of the original PDF
evolution module as it inherits from it many of the setting functions
that we will describe in the next section. We will start describing
the functions of the PDF evolution module and we will devote the
following section to a thourough description of the DIS module.

\section{The PDF evolution module}

The basic usage of the PDF evolution module of {\tt APFEL} requires
only two steps to have the complete set of evolved PDFs. The first
step is the initialization of {\tt APFEL} through the call of the
following function:
\begin{lstlisting}
  APFEL::InitializeAPFEL();
\end{lstlisting}
This will precompute all the needed evolution operators that enter the
discretized DGLAP equation. Let us recall that once the general
settings of the evolution have been defined (perturbative order, heavy
quark masses, reference value of $\alpha_s$, and so on), the
initialization needs to be performed only once, irrespective of the
scales that are used in the PDF evolution. The second step consists in
performing the actual PDF evolution between the initial scale {\tt Q0}
and the final scale {\tt Q} (in GeV). This can be achieved using the
function:
\begin{lstlisting}
  APFEL::EvolveAPFEL(Q0,Q);
\end{lstlisting}
Calling this function {\tt APFEL} solves the discretized DGLAP
equations using the evolution operators precomputed in the
initialization step.

Now the user can access the evolved PDFs at the scale {\tt Q} via the
use of the functions:
\begin{lstlisting}
   APFEL::xPDF(i,x);
   APFEL::xgamma(x);
\end{lstlisting}
where the real variable {\tt x} is the desired value of Bjorken-$x$
while the integer variable {\tt i} in the function {\tt xPDF}, which
runs from $-$6 to 6, corresponds to the quark flavor index according
to the following convention:
\begin{table}[h]
\centering
\begin{tabular}{rcccccccccccccc}
{\tt i}~:  & $-$6 &$-$5 &$-$4&$-$3&$-$2&$-$1&\;0\;&\;1\;&\;2\:&\;3\;&\;4\;&\;5\;&\;6\;\\ 
{\tt xPDF}~: &  $\bar{t}$&$\bar{b}$&$\bar{c}$&$\bar{s}$&$\bar{u}$&$\bar{d}$&
$g$&$d$&$u$&$s$&$c$&$b$&$t$ \\
\end{tabular}
\end{table}
Note that in {\tt APFEL} we have explicitly separated the access to
the quark and gluon PDFs (via {\tt xPDF}) and from that to the photon
PDF (via {\tt xgamma}). Note also that the functions {\tt xPDF} and
{\tt xgamma} return $x$ times the PDFs ($i.e.$ the momentum
fractions).

The basic information given above is enough to write a simple and yet
complete program that performs PDF evolution using {\tt APFEL}.
%
As an illustration, a {\tt C/C++} program that computes and tabulates
PDFs to be compared with the Les Houches PDF benchmark evolution
tables would be the following:
\begin{lstlisting}
#include <iostream>
#include <iomanip>
#include <cmath>
#include "APFEL/APFEL.h"
using namespace std;

int main()
{
  // Define grid in x
  double xlha[11] = {1e-7, 1e-6, 1e-5, 1e-4, 1e-3, 1e-2, 
		1e-1, 3e-1, 5e-1, 7e-1, 9e-1};
  
  // Precomputes evolution operators on the grid
  APFEL::InitializeAPFEL();

  // Perform evolution
  double Q0 = sqrt(2);
  double Q  = sqrt(10000);
  APFEL::EvolveAPFEL(Q0,Q);

  cout << scientific << setprecision(5) << endl;
  cout << "   x   " 
       << setw(11) << "   u-ubar   " 
       << setw(11) << "   d-dbar   " 
       << setw(11) << " 2(ubr+dbr) " 
       << setw(11) << "   c+cbar   " 
       << setw(11) << "   gluon    " 
       << setw(11) << "   photon   " << endl;

  cout << scientific;
  // Tabulate PDFs for the LHA x values
  for (int i = 0; i < 11; i++)
    cout << xlha[i] << "\t"  
	 << APFEL::xPDF(2,xlha[i]) - APFEL::xPDF(-2,xlha[i]) << "\t"
	 << APFEL::xPDF(1,xlha[i]) - APFEL::xPDF(-1,xlha[i]) << "\t"
	 << 2*(APFEL::xPDF(-1,xlha[i]) + APFEL::xPDF(-2,xlha[i])) << "\t"
	 << APFEL::xPDF(4,xlha[i]) + APFEL::xPDF(-4,xlha[i]) << "\t"
	 << APFEL::xPDF(0,xlha[i]) << "\t"
	 << APFEL::xgamma(xlha[i]) << "\t"
	 << endl;

  return 0;
}
\end{lstlisting}
It should be noticed that this example code uses the default settings
of {\tt APFEL} for the evolution parameters such as: initial scale
PDFs, perturbative order, heavy quark masses, values of the couplings,
etc. In the following we will discuss how the user can customize the
settings for the PDF evolution in {\tt APFEL}.

\subsection{Customization of the PDF evolution}\label{EvolCustom}

The customization of the PDF evolution with {\tt APFEL} can be
achieved using a number of dedicated functions, to be called before
the initialization stage, that is before calling {\tt
  InitializeAPFEL}\footnote{This is not entirely precise because there
  is a number of customization functions that are effective only at
  the evolution level and thus can be called also after {\tt
    InitializeAPFEL} but before {\tt EvolveAPFEL}. We will discuss
  this feature case by case when going through all functions available
  in {\tt APFEL}.}. We will subdivide the available functions into
three cathegories:
\begin{itemize}
\item the \textit{setting functions} which provide the real
  costumization tools. These functions allow the user to change the
  way how the initialization and the output (see next items) functions
  behave.
\item The \textit{initialization functions} which are resposible to
  perform the ``main'' operations, like initializing the evolution
  factor and evolving PDFs.
\item The \textit{output functions} which finally return the result of
  a given set of setting and initialization functions.
\end{itemize}
In the following we will list and comment all the functions belonging
to each of the cathegories above. Finally, we remind the reader
that running the configuration script {\tt apfel-config} with the {\tt
  --list-funcs} flag will list all the functions available in {\tt
  APFEL} along with a short explanation.

\subsubsection{Setting functions}

Bofore going through the various setting functions, the user should be
aware of the fact that {\tt APFEL} has a set of default settings that
are used if the user does not intervene to change any of them. This is
why the example described above needs only a very limited number of
steps. However, while each time that {\tt APFEL} is run a banner with
a list of the main settings is displayed, it is usefull to report here
the default settings of {\tt APFEL}. We will first go through all the
setting functions and only at the end we will report the default
settings so that the meaning of all of them should be clear to the
reader who went through this section. 

Bofore proceeding with the descitpion of the setting functions of the
evolution module, it is useful to notice that in the following we will
use the identifiers {\tt int}, {\tt double}, {\tt bool} and {\tt
  string} to specify the type of entry expected by each of the
fuctions described below.

\begin{lstlisting}
  APFEL::SetPerturbativeOrder(int pto);
\end{lstlisting}
This function sets the perturbative order of the PDF evolution to {\tt
  pto}. The integer {\tt pto} can take the values 0, 1 ot 2
according to whether the PDF evolution is performed at
$\mathcal{O}(\alpha_s)$, $\mathcal{O}(\alpha_s^2)$, or
$\mathcal{O}(\alpha_s^3)$, that is LO, NLO and NNLO,
respectively. This function also sets the perturbative order of the
evolution of the couplings $\alpha_s$ and $\alpha$ and possibly of the
heavy quark masses. The default for {\tt pto} is 2.
\begin{lstlisting}
  APFEL::SetTheory(string theory);
\end{lstlisting}
This function sets the theory to be used in the evolution. The
alternatives for {\tt theory} are:
\begin{itemize}
\item {\tt "QCD"}: the PDF evolution is done solving the pure QCD
  DGLAP equations,
\item {\tt "QED"}: the PDF evolution is done solving the pure QED
  DGLAP equations,
\item {\tt "QUniD"}: the PDF evolution is done solving the coupled
  QCD+QED DGLAP equations as explained above.
\end{itemize}
There are more options available that access more ``exotic'' (and
obsolete) solutions of the coupled QCD+QED DGLAP equations. They are:
\begin{itemize}
\item {\tt "QCEDP"} for QCD+QED in parallel,
\item {\tt "QCEDS"} for QCD+QED in series,
\item {\tt "QECDP"} for QED+QCD in parallel,
\item {\tt "QECDS"} for QED+QCD in series,
\item {\tt "QavDP"} for the averaged solution in parallel,
\item {\tt "QavDS"} for the averaged solution in series,
\end{itemize}
and they refer to different combinations of the separate QCD and QED
evolutions. The reader can refer to the original {\tt APFEL}
publication for a detailed explanation. However, the use of these
solution is discouraged unless the user is well aware of their
meaning. The default is {\tt "QCD"}.
\begin{lstlisting}
  APFEL::SetVFNS();
\end{lstlisting}
This function sets the Variable-Flavour Number Scheme (VFNS) for the
PDF, $\alpha_s$, and $\alpha$\footnote{In case the
  $\overline{\mbox{MS}}$ definition for the heavy quark masses is used
  (see below) and the running of the masses has been enabled, this
  function sets the VFNS also for the running of the heavy quark
  masses.} evolution. In practice this means that, if any heavy quark
threshold is encountered during the evolution, the solutions of the
DGLAP equation below and above the threshold itself will be properly
matched. This option is used as a default.
\begin{lstlisting}
  APFEL::SetFFNS(int nfl);
\end{lstlisting}
This function, as opposed to {\tt SetVFNS}, sets the Fixed-Flavour
Number Scheme (FFNS) with {\tt nfl} active flavours for the PDF,
$\alpha_s$ and $\alpha$ (and $\overline{\mbox{MS}}$) evolution. This
function forces the evolution to be done with {\tt nfl} active
flavours in any evolution range. The allowed values are {\tt nfl} = 3,
4, 5 and 6.
\begin{lstlisting}
  APFEL::SetAlphaQCDRef(double alpharef, double Qref);
\end{lstlisting}
This function sets the reference values of the strong coupling
$\alpha_s$ at the scale {\tt Qref} in GeV to {\tt alpharef}. The
default is {\tt alpharef} = 0.35 at {\tt Qref} = $\sqrt{2}$ GeV.
\begin{lstlisting}
  APFEL::SetAlphaQEDRef(double alpharef, double Qref);
\end{lstlisting}
This function sets the reference values of the QED coupling $\alpha$
at the scale {\tt Qref} in GeV to {\tt alpharef}. The default is
{\tt alpharef} = 7.496252$\cdot 10^{-3}$ at {\tt Qref} = 1.777
GeV.
\begin{lstlisting}
  APFEL::SetLambdaQCDRef(double lambdaref, int nref);
\end{lstlisting}
This function sets the value of $\Lambda_{\rm QCD}$ in GeV with {\tt
  nref} flavours to {\tt lambdaref}. This value is used only if
the {\tt lambda} solution of the $\beta$-function equation (see below)
is used to compute the running of $\alpha_s$. The default is {\tt
  lambdaref} = 0.220 GeV with {\tt nref} = 5.
\begin{lstlisting}
  APFEL::SetPoleMasses(double mc, double mb, double mt);
\end{lstlisting}
This function sets the values of the heavy quark thresholds in GeV and
sets the pole-mass scheme as a renormalization scheme for the heavy
quark masses. This function is used as a default with {\tt mc} =
$\sqrt{2}$ GeV, {\tt mb} = 4.5 GeV, {\tt mt} = 175 GeV.
\begin{lstlisting}
  APFEL::SetMSbarMasses(double mc, double mb, double mt);
     sets the values of the heavy quark thresholds
     in GeV in the MSbar scheme.
\end{lstlisting}
This function, as opposed to {\tt SetPoleMasses}, sets the values of
the heavy quark masses in GeV and sets the $\overline{\mbox{MS}}$
scheme as a renormalization scheme for the heavy quark masses. The
reference scales at which the masses are defined can be specified
using the {\tt SetMassScaleReference} function (see below).
\begin{lstlisting}
  APFEL::SetMassScaleReference(double Qc, double Qb, double Qt);
\end{lstlisting}
This function sets the reference scales in GeV at which heavy quark
masses are given. This function is effective only if the
$\overline{\mbox{MS}}$ definition for the heavy quark masses is used
and has no effect if the pole masses are used. If this function is not
called, {\tt APFEL} will assume that the mass reference scales are
equal to the masses themselves. In other words, in the absence of a
call to this function when using the $\overline{\mbox{MS}}$ definition
for the heavy quark masses, the {\tt SetMSbarMasses} function will
define $m_c(m_c)$, $m_b(m_b)$ and $m_t(m_t)$. If instead this function
is called and the reference scales are found to be different from the
masses themselves, the values of $m_c(m_c)$, $m_b(m_b)$ and $m_t(m_t)$
are also evaluated by applying the RG evolution as they are needed as
thresholds for the VFNS evolution when using the
$\overline{\mbox{MS}}$ definition for the heavy quark masses.
\begin{lstlisting}
  APFEL::EnableMassRunning(bool);
\end{lstlisting}
This function enables or disables the running of the
$\overline{\mbox{MS}}$ masses.  This is effective only if the {\tt
  SetMSbarMasses} is also called and in practice switches on and off
the solution RG equation of the heavy quark masses. The default is
{\tt true}.
\begin{lstlisting}
  APFEL::SetTauMass(double mtau);
\end{lstlisting}
This function sets the values of the $\tau$ lepton in GeV to {\tt
  mtau}. This function is effective only if the evolution of the
lepton PDFs is enabled (see below). The default is {\tt mtau} =
1.777 GeV.
\begin{lstlisting}
  APFEL::SetMaxFlavourAlpha(int nf);
\end{lstlisting}
This function sets the maximum number of active flavours in the
evolution of the couplings $\alpha_s$ and $\alpha$ (and the masses) to
{\tt nf}. In practice, this function forces the code not to match the
solution of the $\beta$-function and $\gamma$-function equations at a
given threshold if that threshold is above the maximum number allowed
{\tt nf}. The default is {\tt nf} = 6.
\begin{lstlisting}
  APFEL::SetMaxFlavourPDFs(int nf);
\end{lstlisting}
This function sets the maximum number of active flavours in the
evolution of PDFs to {\tt nf}. In practice, this function forces the
code not to match the solution of the DGLAP equation at a given
threshold if that threshold is above the maximum number allowed {\tt
  nf}. The default is {\tt nf} = 6.
\begin{lstlisting}
  APFEL::SetRenFacRatio(double ratio);
\end{lstlisting}
This function sets the ratio between factorization scale $\mu_F$
(entering PDFs) and renormalization scale $\mu_R$ (entering the
couplings and possibly the heavy quark masses) to {\tt ratio}. If
{\tt ratio} is different from one, {\tt APFEL} will assume that
$\mu_R$ = $\mu_F /$ {\tt ratio} and, as explained above, this gives
rise to additional terms in the higher order splitting functions. The
default is {\tt ratio} = 1.
\begin{lstlisting}
  APFEL::SetTimeLikeEvolution(bool);
\end{lstlisting}
This function enables or disables the time-like evolution. This
evolution, as opposed to the space-like evolution used for PDFs, is
used to evolve fragmentation functions (FFs). The default is {\tt
  false}.
\begin{lstlisting}
  APFEL::SetSmallxResummation(bool, string la);
\end{lstlisting}
This function enables or disables the small-$x$ resummation in the
evolution and set the logarithmic accuracy of the resummation to {\tt
  la}. The possible options for {\tt la} are {\tt "LL"} and {\tt
  "NLL"}. The small-$x$ resummation of the evolution relies on an
  external code called {\tt HELL} that returns the difference between
  the fixed-order and the resummed splitting functions as explained
  above. By default, the small-$x$ resummation is disabled.
\begin{lstlisting}
  APFEL::SetAlphaEvolution(string evol);
\end{lstlisting}
This function sets the solution of the $\beta$-function equations for
the running couplings to {\tt evol}. The variable {\tt evol} can
take the following strings:
\begin{itemize}
\item {\tt "exact"}: the $\beta$-function equations are solved
  numerically in an exact way using the Runge-Kutta method. The
  boundary condition is given by the reference values of the coupling
  at the reference scales.
\item {\tt "expanded"}: the $\beta$-function equations are solved
  analytically by expanding the inverse of the $\beta$-function when
  computing the solution of the RG equation. See above for more
  details. The boundary condition is given by the reference values of
  the coupling at the reference scales.
\item {\tt "lambda"}: the $\beta$-function equations are solved in an
  analytical way in terms of the Landau pole $\Lambda_{\rm QCD}$. See
  above for more details.
\end{itemize}
It should be noticed that a different choice of {\tt evol} only
affects the running of $\alpha_s$ beyond LO while the running of
$\alpha$, being computed always at LO, is left unchanged. The default
is {\tt evol} = {\tt "exact"}.
\begin{lstlisting}
  APFEL::SetPDFEvolution(string evolp);
\end{lstlisting}
This function sets the solution of the DGLAP equations for the
evolution of PDFs to {\tt evolp}. The variable {\tt evolp} can
take the following strings:
\begin{itemize}
\item {\tt "exactmu"}: the DGLAP equation differential in the
  factorization scale $\mu_F$ is solved numerically in an exact way
  using the Runge-Kutta method.
\item {\tt "exactalpha"}: the DGLAP equation differential in the
  coupling $\alpha_s$ is solved numerically in an exact way using the
  Runge-Kutta method. This solution is completely equivalent to {\tt
    exactmu}.
\item {\tt "expandalpha"}: the DGLAP equation differential in the
  coupling $\alpha_s$ is solved analytically by expanding the ratio
  between splitting functions and $\beta$-function. See above for more
  details.
\item {\tt "truncated"}: this solution mimics the $N$-space truncated
  solution and its implementation requires the numerical derivatives
  of the {\tt expandalpha} solution. A detailed explanantion of the
  implementation of thsi particular solution is given above.
\end{itemize}
In all cases the boundary conditions are given by the input initial
scale PDFs. The default is {\tt evolp} = {\tt "exactmu"}.
\begin{lstlisting}
  APFEL::SetEpsilonTruncation(double eps);
\end{lstlisting}
If the {\tt truncated} evolution for PDFs has been chosen by calling
the {\tt SetPDFEvolution} function, the {\tt SetEpsilonTruncation}
function sets the truncation parameter $\epsilon$ used to compute the
numerical dericatives to {\tt eps}.
\begin{lstlisting}
  APFEL::SetPDFSet(string name);
\end{lstlisting}
This function sets the PDF set to be evolved to {\tt name}. The
string variable {\tt name} can be the name of an LHAPDF set. This is
needed to distinguish the LHAPDF sets from the other possible options
available. Other possible options for {\tt name} are:
\begin{itemize}
\item {\tt "ToyLH"}: this option returns the so-called toy LH PDFs
  that have been conceived for benchmark purposes. Details can be
  found in Ref.~???.
\item {\tt "external"}: this option is useful to evolve an
  ``external'' set of PDFs provided by the user. If this option is
  chosen, {\tt APFEL} will look for a routine called {\tt
    ExternalSetAPFEL} whose structure is the following:
\begin{center}
\tt subroutine ExternalSetAPFEL(x,Q0,xf)
\end{center}
 where {\tt x} and {\tt Q0} are the input values of the Bjorken
 variable and the scale in GeV, while {\tt xf(-6:7)} is the output
 array whose entries from {\tt $-$6} to {\tt 6} are the usual quark
 and gluon distributions (times {\tt x}) and the {\tt 7}-th entry
 corresponds instead to the photon PDF.
\item {\tt "external1"}: This option is completely analogous to {\tt
  external} with the only difference that the name of the routined
  expected is now {\tt ExternalSetAPFEL1}. A second slot for an
  external set might be useful in some cases if the user want to
  evolve a second PDF set in the same code.
\item {\tt "repexternal"}: This option is an extension of the previous
  two and the only difference is that the code expects a soubroutine
  called {\tt ExternalSetAPFELRep} whose structure is:
\begin{center}
\tt subroutine ExternalSetAPFELRep(x,Q0,irep,xf)
\end{center}
where, in addition to the entries discussed above, there is the
integer {\tt irep} that identifies the {\tt irep}-th replica of the
input set.  This option might be useful is the input set has more than
one replica but it's not directly an {\tt LHAPDF} set.
\item {\tt "leptexternal"}: This option is a further extension of the
  original {\tt external} option in that it also provides as an output
  the lepton PDFs. In particular, if this option is chosen, the code
  looks for a subroutine called {\tt ExternalSetAPFELLept} whose
  structure is the following:
\begin{center}
\tt subroutine ExternalSetAPFELLept(x,Q0,irep,xf,xl)
\end{center}
where there is the additional output array {\tt xl(-3:3)} whose
entries correspond to:
\begin{center}
{\tt xl(-3)} = $x\tau^+(x,Q_0)$,\\ 
{\tt xl(-2)} = $x\mu^+(x,Q_0)$,\\
{\tt xl(-1)} = $xe^+(x,Q_0)$,\\
{\tt xl(0)}  = $x\gamma(x,Q_0)$,\\
{\tt xl(1)}  = $xe^-(x,Q_0)$,\\
{\tt xl(2)}  = $x\mu^-(x,Q_0)$,\\
{\tt xl(3)}  = $x\tau^-(x,Q_0)$.
\end{center}
In addition, as compared to the options above, the index of the array
{\tt xf} now runs from {\tt -6} to {\tt 6} (and no longer to {\tt 7})
as the photon PDF is now the {\tt 0}-th entry of the array {\tt xl}.
\item {\tt "pretabulated"}: This option is mainly conceived for an
  internal purposes but the user the expert user might want to exploit
  it becuase it allows one to evolve PDF sets whose functional form is
  not know as it makes possible to evolve PDFs starting from their
  value in a finite number of points. However, it is necessary to know
  the full structure of the internal grids used by {\tt APFEL}. This
  information is anyway (partially) accessible using some \textit{ad
    hoc} functions of {\tt APFEL}.In particular, if this option is
  chosen, the codes looks for a subroutine called {\tt
    pretabulatedPDFs} with the following structure:
\begin{center}
\tt pretabulatedPDFs(igrid,alpha,xf,xl)
\end{center}
where the integer {\tt igrid} identifies the subgrid and the integer
{\tt alpha} runs over the nodes of the {\tt igrid}-th grid. Again,
{\tt xf(-6:6)} and {\tt xl(-3:3)} are the quark and gluon, and the
lepton and photon PDFs.
\item {\tt "pretabulated1"}: This option is a copy of {\tt
  pretabulated} and the only difference is that in this case the code
  looks for a subroutine called {\tt pretabulatedPDFs1} have the very
  same structure of that shown above.
\item {\tt "apfel"}: This option is again conceived for internal
  purposes but it might turn out to be useful if a higher performance
  of the evolution is required. If the {\tt apfel} option is chosen,
  the code will use as initial PDFs the PDFs evolved from a previous
  call of the function {\tt EvolveAPFEL}. Of course, this requires
  that a previous evolution with a PDF set different from {\tt apfel}
  exists. This might be convenient when one want to tabulate PDFs over
  a grid in $Q$ in a faster way. In fact, using the {\tt apfel} option
  allows one to split a wide range in $Q$ into small sequential steps
  without the need to do every single evolution starting from the same
  initial scale. This is convenient because the execution time is
  proportional to the wideness of the energy range covered by the
  evolution.
\end{itemize}
There are other options that are strictly used for internal purposes
and thus there is no need to list them here. The default is {\tt
  name} = {\tt "ToyLH"}.
\begin{lstlisting}
  APFEL::SetReplica(int nr);
\end{lstlisting}
This function selects the member or replica of the PDF set to be
evolved to {\tt nr}. This function is effective only if an LHAPDF
set is used and, of course, the integer {\tt nr} must be within zero
and the maximum number of members/replicas contained in the LHAPDF set
in use. The default is default {\tt nr} = 0, that is, according to
the LHAPDF convention, the central member.
\begin{lstlisting}
  APFEL::SetQLimits(double Qmin, double Qmax);
\end{lstlisting}
This function sets the energy range in GeV within which the evolution
is allowed. In practice, for a given pair of {\tt Qmin} and {\tt
  Qmax}, the initial scale {\tt Q0} and the final scale {\tt Q} of any
evolution must be such that {\tt Qmin < Q0, Q < Qmax}. The default is
{\tt Qmin} = 0.5 GeV and {\tt Qmax} = 100000 GeV.
\begin{lstlisting}
  APFEL::SetNumberOfGrids(int ng);
\end{lstlisting}
This function sets the number of internal $x$-space subgrids to be
used by {\tt APFEL} to {\tt ng}. The parameters of the single grids
must be specified grid by grid using the {\tt SetGridParameters}
function discussed below. The default is {\tt ng} = 3.
\begin{lstlisting}
  APFEL::SetGridParameters(int i, int n, int deg, double x);
\end{lstlisting}
This function must always be associated with one call to the {\tt
  SetNumberOfGrids} function described above. It sets the parameter of
the {\tt i}-th subgrid where {\tt i} must run from 1 to {\tt ng} and
thus there must be exactly {\tt ng} calls to {\tt SetGridParameters}.
{\tt n} corresponds to the number intervals (not nodes) of the {\tt
  i}-th subgrid, {\tt deg} identifies the degree of the lagrange
polynomials and thus it is bound to be bigger than zero, and finally
{\tt x} is lower bound of the grid being the upper bound always
assumed to be 1. It is necessary that the value of {\tt x} is
increasingly bigger as the index {\tt igrid} increases. In practice
this mean that the lower bound of the {\tt igrid}-th subgrid must be
in the range covered by the ({\tt igrid} - 1)-th subgrid.
\begin{lstlisting}
  APFEL::SetExternalGrid(int i, int np, int deg, double *x);
     sets the external grid in the position i with
     np intervals, interpolation degree deg. x
     must be a one-dimentional array with upper bound
     in 1 (there cannot be more than 1 external grid).
\end{lstlisting}
\begin{lstlisting}
  APFEL::SetFastEvolution(bool);
     sets the fast PDF evolution (default true).
\end{lstlisting}
\begin{lstlisting}
  APFEL::GetVersion();
     returns the APFEL version in use.
\end{lstlisting}
\begin{lstlisting}
  APFEL::EnableWelcomeMessage(bool);
     enables the printing of the welcome message with
     the APFEL banner and the report of the evolution
     parameters (default true).
\end{lstlisting}
\begin{lstlisting}
  APFEL::EnableEvolutionOperator(bool);
     enables the computation of the external evolution
     parameters (default false).
\end{lstlisting}
\begin{lstlisting}
  APFEL::EnableLeptonEvolution(bool);
     enables the evolution of the lepton PDFs when the
     fast QUniD is used (default false).
\end{lstlisting}
\begin{lstlisting}
  APFEL::LockGrids(bool);
     locks the subgrids (default false).
\end{lstlisting}
\begin{lstlisting}
  APFEL::CleanUp();
     resets all the evolution parameters to the
     default settings.
\end{lstlisting}
\begin{lstlisting}
  APFEL::SetLHgridParameters(int nx, int nxm, double xmin, double xm, double xmax, int nq2, double q2min, double q2max);
     sets the parameters of the grid over which PDFs
     will be tabulated in the LHAPDF format.
\end{lstlisting}
\begin{lstlisting}
  APFEL::ListFunctions();
     lists all the functions available in APFEL.
\end{lstlisting}




\begin{lstlisting}
 Initialization functions: 
  APFEL::InitializeAPFEL();
     initializes the APFEL library. If no settings has
     been specified, it uses the default ones.
  APFEL::EvolveAPFEL(double Q0, double Q);
     evolves PDFs on the grid to the scale Q [GeV]
     starting from the scale Q0 [GeV].
  APFEL::DeriveAPFEL(double Q);
     computes the logarithmic derivative with respect
     of Q of PDFs at the scale Q [GeV].
\end{lstlisting}





\begin{lstlisting}    
 Output functions:
  APFEL::xPDF(int i, double x) and xgamma(double x);
     return "x" times the i-th and the photon PDF
     in "x" at the final scale "Q" [GeV] defined in
     "EvolveAPFEL".
  APFEL::xPDFall(double x, double *xf);
     returns at once "x" times all the PDF in the
     array xf[-6:6] computed in "x" at the final scale
     "Q" [GeV] defined in "EvolveAPFEL".
  APFEL::xPDFj(int i, double x) and xgammaj(double x);
     return "x" times the i-th and the photon PDF
     in "x" at the final scale "Q" [GeV] defined in
     "EvolveAPFEL" interpolated on the joint grid.
  APFEL::dxPDF(int i, double x) and dxgamma(double x);
     return "x" times the derivative in ln(Q^2) of
     the i-th and the photon PDF in "x" at the scale
     "Q" [GeV] defined in "DeriveAPFEL".
  APFEL::NPDF(int i, int N) and Ngamma(int N);
     return the N-th moment of the i-th and the
     photon PDF the final scale "Q" [GeV] defined in
     "EvolveAPFEL".
  APFEL::LUMI(int i, int j, double S);
     returns the partonic luminosity of the i-th and
     j-th partons for the CoM energy S [GeV^2] for the
     final invariant mass Mx = Q [GeV] defined in 
     "EvolveAPFEL".
  APFEL::AlphaQCD(double Q);
     returns the QCD strong coupling alpha_s at the
     scale "Q" [GeV].
  APFEL::AlphaQED(double Q);
     returns the QED coupling alpha at the scale
     "Q" [GeV].
  APFEL::HeavyQuarkMass(int i,double Q);
     returns the mass of the i-th heavy quark
     (i = 4,5,6) scale "Q" [GeV] (the masses run only
     when using the MSbar scheme).
  APFEL::nIntervals();
     returns the number of intervals of the joint
     grid.
  APFEL::xGrid(int alpha);
     returns the value of "x" on the alpha-th node of
     the joint grid.
  APFEL::GetPerturbativeOrder();
     returns the perturbative order set for the
     evolution
  APFEL::ExternalEvolutionOperator(string fname, int i, int j, double x, int beta);
     returns the PDF evolution operator.
  APFEL::LHAPDFgrid(int Nrep, double Qin, string fname);
     produces a PDF interpolation grid in the LHAPDF
     format.
  APFEL::LHAPDFgridDerivative(int Nrep, string fname);
     produces an interpolation grid in the LHAPDF
     format for the derived PDFs.
\end{lstlisting}



\begin{lstlisting}
 ---- Functions of the DIS module ----
 
 Initialization functions:
 
  APFEL::InitializeAPFEL_DIS();
     initializes the DIS module. If no settings has
     been specified, it uses the default ones.
  APFEL::ComputeStructureFunctionsAPFEL(double Q0, double Q);
     computes the DIS structure functions on the grid
     at the scale "Q" [GeV] applying also the PDF
     evolution from the initial scale "Q0" [GeV].
   
 Setting functions:
 
  APFEL::SetMassScheme(string ms);
     sets the mass scheme to be used to compute the
     structure functions ("ms" = "ZM-VFNS", "FFNS",
     "FONLL-A", "FONLL-B", "FONLL-C", default "ms" =
     "ZM-VFNS").
  APFEL::SetPolarizationDIS(double pol);
     sets the beam polarization (default "pol" = 0).
  APFEL::SetProcessDIS(string pr);
     sets process ("pr" = "EM", "NC", "CC", default
     "pr" = "EM").
  APFEL::SetProjectileDIS(string lept);
     sets the projectile ("lept" = "electron",
     "positron", "neutrino", "antineutrino", default
     "lept" = "electron").
  APFEL::SetTargetDIS(string tar);
     sets the target ("tar" = "proton", "neutron",
     "isoscalar", "iron", default "tar" = "proton")
  APFEL::SetZMass(double massz);
     sets the value of the mass of the Z boson
     (default "massz" = 91.1876 GeV).
  APFEL::SetWMass(double massw);
     sets the value of the mass of the W boson
     (default "massw" = 80.385 GeV).
  APFEL::SetProtonMass(double massp);
     sets the value of the mass of the proton
     (default "massp" = 0.938272046 GeV).
  APFEL::SetSin2ThetaW(double sw);
     sets the value of sin^2(theta_W)
     (default "sw" = 0.23126).
  APFEL::SetGFermi(double gf);
     sets the value of Fermi constant
     (default "gf" = 1.1663787e-5).
  APFEL::SetCKM(double vud, double vus, double vub,
          double vcd, double vcs, double vcb,
          double vtd, double vts, double vtb);
     sets the absolute value of the entries of the
     CKM matrix
     (default: 0.97427d0, 0.22536d0, 0.00355d0,
               0.22522d0, 0.97343d0, 0.04140d0,
               0.00886d0, 0.04050d0, 0.99914d0).
  APFEL::SetRenQRatio(double ratio);
     sets the ratio muR / Q (default 1)
  APFEL::SetFacQRatio(double ratio);
     sets the ratio muF / Q (default 1)
  APFEL::EnableDynamicalScaleVariations(bool);
     enables or disables the possibility to perform
     fact/ren scale variations point by point without
     requiring the ratio \mu_{R,F} / Q to be constant.
     Limitations: \mu_F = \mu_R and slower code.
  APFEL::EnableTargetMassCorrections(bool);
     enables or disables the target mass corrections
     to the DIS structure functions due to the finite
     mass of the proton.
  APFEL::EnableDampingFONLL(bool);
     enables or disables the damping factor when the
     FONLL structure functions are computed.
  APFEL::SelectCharge(string selch);
     selects one particular charge in the NC structure
     functions ("selch" = "down", "up", "strange",
     "charm", "bottom", "top", "all", default 
     "selch" = "all")
  APFEL::SetPropagatorCorrection(double dr);
     sets the correction to the Z propagator involved
     in the NC DIS structure functions
     (default "dr" = 0).
  APFEL::SetEWCouplings(double vd, double vu, double ad, double au);
     sets the vector and axial couplings of the up-
     and down-type quarks. If they are not set by the
     user the standard couplinglings are used.
   
 Output functions:
 
  APFEL::F2light(double x), F2charm(double x), F2bottom(double x),
   F2top(double x), F2total(double x);
  APFEL::FLlight(double x), FLcharm(double x), FLbottom(double x),
   FLtop(double x), FLtotal(double x);
  APFEL::F3light(double x), F3charm(double x), F3bottom(double x),
   F3top(double x), F3total(double x);
     return the F2, FL and xF3 struncture functions in
     "x" at the final scale "Q" [GeV] defined in
     "ComputeStructureFunctionsAPFEL".
  APFEL::GetZMass();
     returns the value of the mass of the Z boson
  APFEL::GetWMass();
     returns the value of the mass of the W boson
  APFEL::GetProtonMass();
     returns the value of the mass of the proton
  APFEL::GetSin2ThetaW();
     returns the value of sin^2(theta_W)
  APFEL::GetGFermi();
     returns the value of Fermi constant
  APFEL::GetCKM(int u, int d);
     returns the absolute value of the (u,d) entry
     of the CKM matrix
  APFEL::GetSIATotalCrossSection(int pto, double Q);
     returns the SIA total cross section in natural
     units at the perturbative order "pto" and at the
     scale "Q" in GeV (only for time-like evolution).
  APFEL::ExternalDISOperator(string SF, int ihq, int i, double x, int beta);
     returns the DIS operators.
   
 Special functions for the production of FK tables:
   
  APFEL::SetFKObservable(string obs);
  APFEL::GetFKObservable();
  APFEL::FKSimulator(string obs, double x, double q, double y, int i, int beta);
  APFEL::FKObservables(string obs, double x, double q, double y);
  APFEL::ComputeHardCrossSectionsDY(string inputfile, string outputfile);
  APFEL::ComputeFKTables(string inputfile, double Q0, int flmap[196]);
\end{lstlisting}



Using the dafualt settings of {\tt APFEL} this is how the evolution
banner would look like:
\begin{lstlisting}
 Welcome to 
      _/_/_/    _/_/_/_/   _/_/_/_/   _/_/_/_/   _/
    _/    _/   _/    _/   _/         _/         _/
   _/_/_/_/   _/_/_/_/   _/_/_/     _/_/_/     _/
  _/    _/   _/         _/         _/         _/
 _/    _/   _/         _/         _/_/_/_/   _/_/_/_/
 _____v2.6.1 A PDF Evolution Library, arXiv:1310.1394
      Authors: V. Bertone, S. Carrazza, J. Rojo
 
 Report of the evolution parameters:
   
 QCD evolution
 Space-like evolution (PDFs)
 Evolution scheme: VFNS at N2LO
 Solution of the DGLAP equation: "exactmu" with maximum 6 active flavours
 Solution of the coupling equations: "exact" with maximum 6 active flavours
 Coupling reference value:
 - AlphaQCD(  1.4142 GeV) =  0.350000
 Pole heavy quark thresholds:
 - Mc =   1.414 GeV
 - Mb =   4.500 GeV
 - Mt = 175.000 GeV
 muR / muF =  1.0000
  
 Allowed evolution range [   0.50 : 100000.00 ] GeV
 Fast evolution enabled
\end{lstlisting}
As clear, the most important settings are reported in the banner and
we recommand to consult the banner every time that {\tt APFEL} is run
to make sure that the desired setting are actually used.

Here is the complete set of default settings used by {\tt APFEL}:
\begin{lstlisting}
  APFEL::EnableWelcomeMessage(true);
  APFEL::SetQLimits(0.5,100000);
  APFEL::SetPerturbativeOrder(2);
  APFEL::SetVFNS
  APFEL::SetTheory("QCD");
  APFEL::SetFastEvolution(true);
  APFEL::SetTimeLikeEvolution(false);
  APFEL::SetSmallxResummation(false,"NLL");
  APFEL::SetAlphaQCDRef(0.35,sqrt(2););
  APFEL::SetAlphaQEDRef(7.496252e-3,1.777);
  APFEL::SetLambdaQCDRef(0.220,5);
  APFEL::SetEpsilonTruncation(1e-5);
  APFEL::SetAlphaEvolution("exact");
  APFEL::SetPDFEvolution("exactmu");
  APFEL::SetRenFacRatio(1);
  APFEL::SetPoleMasses(sqrt(2),4.5,175);
  APFEL::SetMassScaleReference(sqrt(2),4.5,175);
  APFEL::SetTauMass(1.777);
  APFEL::EnableMassRunning(true);
  APFEL::SetMaxFlavourPDFs(6);
  APFEL::SetMaxFlavourAlpha(6);
  APFEL::SetPDFset("ToyLH");
  APFEL::SetReplica(0);
  APFEL::EnableEvolutionOperator(false);
  APFEL::EnableLeptonEvolution(false);
  APFEL::LockGrids(false);
  APFEL::SetLHgridParameters(100,50,1e-9,1e-1,1,50,1,1e10);
  APFEL::SetNumberOfGrids(3);
  APFEL::SetGridParameters(1,80,3,1e-5);
  APFEL::SetGridParameters(2,50,5,1e-1);
  APFEL::SetGridParameters(3,40,5,8e-1);
\end{lstlisting}


As an illustration, if the user wants to perform the QCD evolution at NLO
instead of  the default NNLO, she/he needs to add to the code above,
before the initialization routine {\tt InitializeAPFEL}, a call to the
corresponding function, that is:
\begin{lstlisting}
 APFEL::SetPerturbativeOrder(1);
\end{lstlisting}
or if the user wants to use as a boundary condition for the PDF
evolution a particular set available through the {\tt LHAPDF} interface, say
{\tt NNPDF23\_nlo\_as\_0118\_qed.LHgrid}, she/he needs
to call before the initialization the following function:
\begin{lstlisting}
 APFEL::SetPDFSet("NNPDF23_nlo_as_0118_qed.LHgrid");
\end{lstlisting}
By default, {\tt APFEL} will use the central replica of
the selected PDF set. 
%
Varying any other setting is similar, various example programs
have been collected in the {\tt examples} folder in the {\tt APFEL} source
folder.
%

When modifying the default settings, particular care 
must be taken with the number of interpolation grids, the number
of points in each grid and the order of the interpolation.
%
The  default
settings in {\tt APFEL} use three grids whose ranges and
number of points have been tuned to give accurate and
fast results over a wide range of $x$.
%
If the default parameters are modified, the user should check
that the accuracy is still good enough, by comparing for instance
with another run of {\tt APFEL} with the default interpolation
parameters.

The folder {\tt examples} in the {\tt APFEL} source directory contains
several examples that further illustrate the functionalities of
the code, and that can be used by the user as a starting point
towards a program that suits her/his particular physics needs.
%
All these examples are available in the three possible interfaces to
{\tt APFEL}: {\scshape Fortran 77}, {\tt C/C++} and {\tt Python}.


\section{The DIS module}

%%%%%%%%%%%%%%%%%%%%%%%%%%%%%%%%%%%%%%%%%%%%%%%%%%%%%%%%
\clearpage

\bibliography{qedevol}

\end{document}

%%%%%%%%%%%%%%%%%%%%%%%%%%%%%%%%%%%%%%%%%%%%%%%%%%%
%%%%%%%%%%%%%%%%%%%%%%%%%%%%%%%%%%%%%%%%%%%%%%%%%%%
