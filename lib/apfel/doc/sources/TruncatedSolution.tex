%% LyX 2.0.2 created this file.  For more info, see http://www.lyx.org/.
%% Do not edit unless you really know what you are doing.
\documentclass[twoside,english]{article}
\usepackage{lmodern}
\usepackage[T1]{fontenc}
\usepackage[latin9]{inputenc}
\usepackage[a4paper]{geometry}
\geometry{verbose,tmargin=3cm,bmargin=2.5cm,lmargin=2cm,rmargin=2cm}
\usepackage{babel}
\usepackage{float}
\usepackage{bm}
\usepackage{amsmath}
\usepackage{esint}
\usepackage[unicode=true,pdfusetitle,
 bookmarks=true,bookmarksnumbered=false,bookmarksopen=false,
 breaklinks=false,pdfborder={0 0 0},backref=false,colorlinks=false]
 {hyperref}

\makeatletter

%%%%%%%%%%%%%%%%%%%%%%%%%%%%%% LyX specific LaTeX commands.
%% Because html converters don't know tabularnewline
\providecommand{\tabularnewline}{\\}

%%%%%%%%%%%%%%%%%%%%%%%%%%%%%% Textclass specific LaTeX commands.
%\numberwithin{equation}{section}

\makeatother

\begin{document}

\title{Implementation of the Truncated Solution in {\tt APFEL}}

\author{Richard D. Ball and Valerio Bertone}
\maketitle
\begin{abstract}
In this document we describe how to implement the truncated (or
linearized) solution of the DGLAP equation in {\tt APFEL}.
\end{abstract}

\section{The Algorithm}

The DGLAP evolution equations, that govern the evolution with respect
to the factorization scale $\mu$, of the vector of parton distribution
functions $\mathbf{f}$ has the form:
\begin{equation}\label{DGLAPmu}
\mu^2\frac{\partial}{\partial \mu^2} \mathbf{f}(x,\mu) = \mathbf{P}(x,\alpha_s(\mu))\otimes \mathbf{f}(x,\mu)\,,
\end{equation}
where $\mathbf{P}$ is the matrix of splitting functions and where we
have introduced the convolution operator $\otimes$ defined as:
\begin{equation}
f(x)\otimes g(x) = \int_0^1 dy \int_0^1 dz f(y)g(z)\delta(x-yz)\,.
\end{equation}
Now, considering the RG equation for the strong coupling $\alpha_s$:
\begin{equation}\label{RGalpha}
\mu_R^2\frac{\partial}{\partial \mu_R^2} \alpha_s(\mu_R) = \beta(\alpha_s(\mu_R))\,,
\end{equation}
which governes the evolution of $\alpha_s$ as a function of the
renormalization scale $\mu_R$ and identifying $\mu_R$ with $\mu$, we
can combine eqs.~(\ref{DGLAPmu}) and~(\ref{RGalpha}) obtaining:
\begin{equation}\label{DGLAPalpha}
\frac{\partial}{\partial \alpha_s} \mathbf{f}(x,\mu) = \mathbf{R}(x,\alpha_s(\mu))\otimes \mathbf{f}(x,\mu)\,.
\end{equation}
where we have defined:
\begin{equation}\label{Rdef}
\mathbf{R}(x,\alpha_s(\mu)) = \frac{\mathbf{P}(x,\alpha_s(\mu))}{\beta(\alpha_s(\mu))}\,.
\end{equation}
Of course, solving eq.~(\ref{DGLAPalpha}) is exactly equivalent to
solving eq.~(\ref{DGLAPmu}) and they are referred to as \textbf{exact}
solutions\footnote{{\tt APFEL} implements the solution
to both eqs.~(\ref{DGLAPalpha}) and~(\ref{DGLAPmu}) and the result are
actually identical.}.

At N$^n$LO, the matrix of splitting functions $\mathbf{P}$ and the function
$\beta$ have the following perturbative expansions:
\begin{equation}
\mathbf{P}(x,\alpha_s(\mu)) = \alpha_s(\mu) \sum_{k=0}^{n} \alpha_s^k(\mu) \mathbf{P}^{(k)}(x)
\end{equation}
and:
\begin{equation}
\beta(\alpha_s(\mu)) = - \alpha_s^2(\mu) \sum_{k=0}^{n} \alpha_s^k(\mu) \beta_k\,.
\end{equation}
and thus also the matrix $\mathbf{R}$ defined in eq.~(\ref{Rdef}) can
be expanded as:
\begin{equation}\label{Rexp}
\mathbf{R}(x,\alpha_s(\mu)) = -\frac1{\alpha_s(\mu)\beta_0} \sum_{k=0}^{n} \alpha_s^k(\mu) \mathbf{R}^{(k)}(x)
\end{equation}
where:
\begin{equation}
\begin{array}{l}
\displaystyle \mathbf{R}^{(0)}(x) =
\mathbf{P}^{(0)}(x)\\
\\
\displaystyle \mathbf{R}^{(1)}(x) =
\mathbf{P}^{(1)}(x) - b_1\mathbf{P}^{(0)}(x)\\
\\
\displaystyle \mathbf{R}^{(2)}(x) =
\mathbf{P}^{(2)}(x) - b_1\mathbf{P}^{(1)}(x) + (
b_1^2 - b_2) \mathbf{P}^{(0)}(x)
\end{array}
\end{equation}
where we have defined:
\begin{equation}
b_k = \frac{\beta_k}{\beta_0}\,.
\end{equation}
We will call the solution of eq.~(\ref{DGLAPalpha}) using the expansion of the
matrix $\mathbf{R}$ in eq.~(\ref{Rexp}) \textbf{expanded} solution.

In order to compute the \textbf{truncated} or \textbf{linearized}
solution, we introduce in the expansion in eq.~(\ref{Rexp}) the small
positive parameter $\varepsilon$ in such a way that:
\begin{equation}\label{Rexp}
\mathbf{R}(x,\alpha_s(\mu))\rightarrow
\mathbf{R}(x,\alpha_s(\mu),\varepsilon) = -\frac1{\alpha_s(\mu)\beta_0} \left[
\mathbf{R}^{(0)}(x) + \varepsilon \alpha_s(\mu) \mathbf{R}^{(1)}(x) +
\varepsilon^2 \alpha_s^2(\mu) \mathbf{R}^{(2)}(x) + \dots\right]\,.
\end{equation}
This will clearly introduce a dependence on $\varepsilon$ in the
evolved PDFs so that the solution of the DGLAP equation will be:
\begin{equation}
\mathbf{f} \equiv \mathbf{f}(x,\mu,\varepsilon)\,. 
\end{equation}

Now, in order to construct the linearized solution at the different
orders we need to Taylor expand the solution around $\varepsilon = 0$
and truncated the expansion at the desired order. Provided that
$\varepsilon$ is small enough, this can easily be done numerically
just by taking the following combinations:
\begin{equation}\label{TruncPDFs}
\begin{array}{l}
\displaystyle \mathbf{f}^{\rm LO}(x,\mu) = \mathbf{f}(x,\mu,0)\\
\\
\displaystyle \mathbf{f}^{\rm NLO}(x,\mu) = \mathbf{f}(x,\mu,0)+\frac{\mathbf{f}(x,\mu,\varepsilon)-\mathbf{f}(x,\mu,-\varepsilon)}{2\varepsilon}+\mathcal{O}(\varepsilon^2\mathbf{f}^{(3)})\\
\\
\displaystyle \mathbf{f}^{\rm NNLO}(x,\mu) = \mathbf{f}(x,\mu,0)+\frac{\mathbf{f}(x,\mu,\varepsilon)-\mathbf{f}(x,\mu,-\varepsilon)}{2\varepsilon}+\frac{\mathbf{f}(x,\mu,\varepsilon)-2 \mathbf{f}(x,\mu,0)+\mathbf{f}(x,\mu,-\varepsilon)}{2\varepsilon^2}+\mathcal{O}(\varepsilon^2\mathbf{f}^{(3)})
\end{array}\,,
\end{equation}
where $\mathbf{f}^{(n)}$ is the $n$-th-order derivative of
$\mathbf{f}$ with respect to $\varepsilon$ evaluated in $\varepsilon =
0$. It should be noticed that $\mathbf{f}^{(n)}\propto
\alpha_s^{n+1}$, for $n \geq 0$, and thus the real error in our case
is $\mathcal{O}(\varepsilon^2\alpha_s^4)$. Finally, it should be
pointed out that the order of the perturbative evolution of $\alpha_s$
in each of the solutions above must be consistent with the order to
which the solution has been truncated.

Based on numerical tests, we observe that the procedure is quite
stable over a wide range of values $\varepsilon$. In particular, the
numerical results do not change significantly in the range
$\varepsilon \in [10^{-3}:10^{-9}]$. As a consequence, we have chosen
to set our default value of $\varepsilon$ to $10^{-5}$.

One final remark regarding the implementation of the algorithm
described above when evolving PDFs in the VFNS is necessary. As is
well known, the evolution in the VFNS is performed by matching the
evolutions with different numbers of active flavors at the heavy quark
thresholds using a suitable set of perturbative matching
conditions. In order to ensure that the matching is consistently done
for all contributions, the prescriptions given in
eq.~(\ref{TruncPDFs}) are applicable only if no heavy quark threshold
is crossed during the evolution. To explain what it practically means,
suppose to evolve PDFs between the scales $\mu_0$ and $\mu$ and that
one heavy quark threshold $\mu_h$ is crossed during the evolution. In
this case, starting from the initial scale PDFs $\mathbf{f}(x,\mu_0)$,
the first step it to compute the contributions
$\mathbf{f}(x,\mu_h,0)$, $\mathbf{f}(x,\mu_h,\varepsilon)$ and
$\mathbf{f}(x,\mu_h,-\varepsilon)$. These contributions must then be
combined using one of the prescriptions given in
eq.~(\ref{TruncPDFs}), according to the perturbative order. Finally
the result of the combination $\mathbf{f}^{\rm N^kLO}(x,\mu_h)$ can be
used as an initial condition to compute the contributions
$\mathbf{f}(x,\mu,0)$, $\mathbf{f}(x,\mu,\varepsilon)$ and
$\mathbf{f}(x,\mu,-\varepsilon)$ that can finally be combined into the
final scale PDFs $ \mathbf{f}^{\rm N^kLO}(x,\mu)$, again using
eq.~(\ref{TruncPDFs}). The procedure can be easily generalized for a
larger number of threshold crossings.

In order to ensure a better accuracy, one might employ a more accurate
algorithm for the computation of the derivatives which relies on a larger
number of points around the LO solution(\footnote{Note that in
  eq.~(\ref{TruncPDFs}) we use three points, $i.e.$ $\varepsilon$, $0$
and $-\varepsilon$.}). We can then use five points rather than
three. This clearly ensures a better accuracy, as the error associated
to this algorithm is $\mathcal{O}(\varepsilon^4)$, but the price to
pay is that we need to compute two more evolution as compared to
algorithm presented in eq.~(\ref{TruncPDFs}), namely for
$\pm2\varepsilon$. The formulas to be implemented are:
\begin{equation}\label{TruncPDFs1}
\begin{array}{lcl}
\displaystyle \mathbf{f}^{\rm LO}(x,\mu) &=& \displaystyle \mathbf{f}(x,\mu,0)\\
\\
\displaystyle \mathbf{f}^{\rm NLO}(x,\mu) &=& \displaystyle\mathbf{f}(x,\mu,0)+\frac{-\mathbf{f}(x,\mu,2\varepsilon)+8\mathbf{f}(x,\mu,\varepsilon)-8\mathbf{f}(x,\mu,-\varepsilon)+\mathbf{f}(x,\mu,-2\varepsilon)}{12\varepsilon}+\mathcal{O}(\varepsilon^4\mathbf{f}^{(5)})\\
\\
\displaystyle \mathbf{f}^{\rm NNLO}(x,\mu) &=&
\displaystyle\mathbf{f}(x,\mu,0)
+\frac{-\mathbf{f}(x,\mu,2\varepsilon)+8\mathbf{f}(x,\mu,\varepsilon)-8\mathbf{f}(x,\mu,-\varepsilon)+\mathbf{f}(x,\mu,-2\varepsilon)}{12\varepsilon}\\
\\
&+&\displaystyle\frac{-\mathbf{f}(x,\mu,2\varepsilon)+16\mathbf{f}(x,\mu,\varepsilon)-30 \mathbf{f}(x,\mu,0)+16\mathbf{f}(x,\mu,-\varepsilon)-\mathbf{f}(x,\mu,-2\varepsilon)}{24\varepsilon^2}+\mathcal{O}(\varepsilon^4\mathbf{f}^{(5)})
\end{array}\,.
\end{equation}
Following the same argument given above, we have that the error
associated to the NLO and NNLO solutions is
$\mathcal{O}(\varepsilon^4\alpha_s^6)$.

\end{document}
