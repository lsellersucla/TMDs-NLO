\documentclass[10pt,a4paper]{article}
\usepackage{amsmath,amssymb,bm,makeidx,subfigure}
\usepackage[italian,english]{babel}
\usepackage[center,small]{caption}[2007/01/07]
\usepackage{fancyhdr}
\usepackage{color}

\definecolor{blu}{rgb}{0,0,1}
\definecolor{verde}{rgb}{0,1,0}
\definecolor{rosso}{rgb}{1,0,0}
\definecolor{viola}{rgb}{1,0,1}
\definecolor{arancio}{rgb}{1,0.5,0}
\definecolor{celeste}{rgb}{0,1,1}
\definecolor{rosa}{rgb}{1,0.3,0.5}

\oddsidemargin = 12pt
\topmargin = 0pt
\textwidth = 440pt
\textheight = 650pt

\makeindex

\begin{document}

\begin{center}
\textbf{\Large Definition of CC Light and Heavy Structure Functions}
\end{center}
\tableofcontents
\newpage

\section{$F_2^{\nu(\overline{\nu})}$ and $F_L^{\nu(\overline{\nu})}$}
Writing explicitely the CKM matrix elements, the proton stucture
function $F_2^{\nu,p}$ in the Zero-Mass Variable Flavour Number Scheme
looks like this:
\begin{equation}\label{f2nup}
\begin{array}{rcl}
F_2^{\nu,p} &=& 2x\Big\{ C_{2,q}\otimes\Big[\left(|V_{ud}|^2+|V_{cd}|^2+|V_{td}|^2\right)d\\
            &+& \left(|V_{ud}|^2+|V_{us}|^2+|V_{ub}|^2\right)\overline{u}\\
            &+& \left(|V_{us}|^2+|V_{cs}|^2+|V_{ts}|^2\right)s\\
            &+& \left(|V_{cd}|^2+|V_{cs}|^2+|V_{cb}|^2\right)\overline{c}\\
            &+& \left(|V_{ub}|^2+|V_{cb}|^2+|V_{tb}|^2\right)b\\
            &+& \left(|V_{td}|^2+|V_{ts}|^2+|V_{tb}|^2\right)\overline{t}\\
            &+& c^{CC}_g(N_f)C_{2,q}\otimes g\Big\}
\end{array}
\end{equation}
where:
\begin{equation}
c^{CC}_g(N_f) = 2\sum_{i=u,c,t}\sum_{j=d,s,b}|V_{ij}|^2
\end{equation}
Note that everything we will do in this Section applies also to
$F_L^{\nu,p}$ just substuting $C_{2,q}$ and $C_{2,g}$ with $C_{L,q}$
and $C_{L,g}$, respectively. Therefore, the stucture of the
observables is exactly the same.

Now, we want to split up the above structure function into its light
part plus the stucture functions of the single heavy quarks, that is:
\begin{equation}
F_2^{\nu,p} = F_{2,l}^{\nu,p} + F_{2,c}^{\nu,p} + F_{2,b}^{\nu,p} + F_{2,t}^{\nu,p}
\end{equation}
without having any overlap between the single components.

To this end, we use the CKM matrix elements and we say that:
\begin{enumerate}
\item the light part is composed by the terms of eq. (\ref{f2nup})
  which are proportional to those CKM matrix elements containing
  \textit{only} light flavours (i.e. $u$, $d$ and $s$)
\item the heavy part due to the heavy flavour $h$ is instead given by
  the terms proportional to $V_{hk}$ or $V_{kh}$, where $m_h^2 >
  m_k^2$
\end{enumerate}
Actually, this means dividing the CKM matrix in the following way:
\begin{equation}
V_{CKM} =
\begin{pmatrix}
{\color{rosso}V_{ud}} & {\color{rosso}V_{us}} & {\color{verde}V_{ub}}\\
{\color{blu}V_{cd}} & {\color{blu}V_{cs}} & {\color{verde}V_{cb}}\\
{\color{viola}V_{td}} & {\color{viola}V_{ts}} & {\color{viola}V_{tb}}
\end{pmatrix}
\end{equation}
where the {\color{rosso} red terms} are those which contribute to the
{\color{rosso} light part} of the proton structure functions, the
{\color{blu} blue terms} to the {\color{blu} charm part}, the
{\color{verde} green terms} to the {\color{verde} bottom part} and the
{\color{viola} violet terms} to the {\color{viola} top part}.

Explicitely we have that:
\begin{equation}
\begin{array}{rcl}  
F_{2,l}^{\nu,p} &=& 2x\Big\{C_{2,q}\otimes\Big[|V_{ud}|^2 d +\left(|V_{ud}|^2+|V_{us}|^2\right)\overline{u} + |V_{us}|^2 s\Big]\\
                &+& 2\left(|V_{ud}|^2+|V_{us}|^2\right)C_{2,g}\otimes g\Big\}\\
\\
F_{2,c}^{\nu,p} &=& 2x\Big\{C_{2,q}\otimes\Big[|V_{cd}|^2(d+\overline{c}) + |V_{cs}|^2 (s+\overline{c})\Big]\\
                &+& 2\left(|V_{cd}|^2+|V_{cs}|^2\right)C_{2,g}\otimes g\Big\}\\
\\
F_{2,b}^{\nu,p} &=& 2x\Big\{C_{2,q}\otimes\Big[|V_{ub}|^2 (\overline{u}+b) + |V_{cb}|^2 (\overline{c}+b)\Big]\\
                &+& 2\left(|V_{ub}|^2+|V_{cb}|^2\right)C_{2,g}\otimes g\Big\}\\
\\
F_{2,t}^{\nu,p} &=& 2x\Big\{C_{2,q}\otimes\Big[|V_{td}|^2 (d +\overline{t})+ |V_{ts}|^2(s+\overline{t}) + |V_{tb}|^2(b+\overline{t})\Big]\\
                &+& 2\left(|V_{td}|^2 + |V_{ts}|^2 + |V_{tb}|^2\right)C_{2,g}\otimes g\Big\}
\end{array}
\end{equation}

We can obtain $F_2^{\overline\nu,p}$ directly from $F_2^{\nu,p}$ just
by exchanging each quark with the respective antiquark and
viceversa. So, we get:
\begin{equation}
\begin{array}{rcl}  
F_{2,l}^{\overline\nu,p} &=& 2x\Big\{C_{2,q}\otimes\Big[|V_{ud}|^2\overline{d}+\left(|V_{ud}|^2+|V_{us}|^2\right)u + |V_{us}|^2 \overline{s}\Big]\\
                &+& 2\left(|V_{ud}|^2+|V_{us}|^2\right)C_{2,g}\otimes g\Big\}\\
\\
F_{2,c}^{\overline\nu,p} &=& 2x\Big\{C_{2,q}\otimes\Big[|V_{cd}|^2(\overline{d}+c) + |V_{cs}|^2 (\overline{s}+c)\Big]\\
                &+& 2\left(|V_{cd}|^2+|V_{cs}|^2\right)C_{2,g}\otimes g\Big\}\\
\\
F_{2,b}^{\overline\nu,p} &=& 2x\Big\{C_{2,q}\otimes\Big[|V_{ub}|^2 (u+\overline{b}) + |V_{cb}|^2 (c+\overline{b})\Big]\\
                &+& 2\left(|V_{ub}|^2+|V_{cb}|^2\right)C_{2,g}\otimes g\Big\}\\
\\
F_{2,t}^{\overline\nu,p} &=& 2x\Big\{C_{2,q}\otimes\Big[|V_{td}|^2 (\overline{d} +t)+ |V_{ts}|^2(\overline{s}+t) + |V_{tb}|^2(\overline{b}+t)\Big]\\
                &+& 2\left(|V_{td}|^2 + |V_{ts}|^2 + |V_{tb}|^2\right)C_{2,g}\otimes g\Big\}
\end{array}
\end{equation}

And finally the neutron structure functions can be obtained from the
proton ones just by exchanging $u(\overline{u})\leftrightarrow
d(\overline{d})$. So:
\begin{equation}
\begin{array}{rcl}  
F_{2,l}^{\nu,n} &=& 2x\Big\{C_{2,q}\otimes\Big[|V_{ud}|^2 u +\left(|V_{ud}|^2+|V_{us}|^2\right)\overline{d} + |V_{us}|^2 s\Big]\\
                &+& 2\left(|V_{ud}|^2+|V_{us}|^2\right)C_{2,g}\otimes g\Big\}\\
\\
F_{2,c}^{\nu,n} &=& 2x\Big\{C_{2,q}\otimes\Big[|V_{cd}|^2(u+\overline{c}) + |V_{cs}|^2 (s+\overline{c})\Big]\\
                &+& 2\left(|V_{cd}|^2+|V_{cs}|^2\right)C_{2,g}\otimes g\Big\}\\
\\
F_{2,b}^{\nu,n} &=& 2x\Big\{C_{2,q}\otimes\Big[|V_{ub}|^2 (\overline{d}+b) + |V_{cb}|^2 (\overline{c}+b)\Big]\\
                &+& 2\left(|V_{ub}|^2+|V_{cb}|^2\right)C_{2,g}\otimes g\Big\}\\
\\
F_{2,t}^{\nu,n} &=& 2x\Big\{C_{2,q}\otimes\Big[|V_{td}|^2 (u +\overline{t})+ |V_{ts}|^2(s+\overline{t}) + |V_{tb}|^2(b+\overline{t})\Big]\\
                &+& 2\left(|V_{td}|^2 + |V_{ts}|^2 + |V_{tb}|^2\right)C_{2,g}\otimes g\Big\}
\end{array}
\end{equation}
and:
\begin{equation}
\begin{array}{rcl}  
F_{2,l}^{\overline\nu,n} &=& 2x\Big\{C_{2,q}\otimes\Big[|V_{ud}|^2\overline{u}+\left(|V_{ud}|^2+|V_{us}|^2\right)d + |V_{us}|^2 \overline{s}\Big]\\
                &+& 2\left(|V_{ud}|^2+|V_{us}|^2\right)C_{2,g}\otimes g\Big\}\\
\\
F_{2,c}^{\overline\nu,n} &=& 2x\Big\{C_{2,q}\otimes\Big[|V_{cd}|^2(\overline{u}+c) + |V_{cs}|^2 (\overline{s}+c)\Big]\\
                &+& 2\left(|V_{cd}|^2+|V_{cs}|^2\right)C_{2,g}\otimes g\Big\}\\
\\
F_{2,b}^{\overline\nu,n} &=& 2x\Big\{C_{2,q}\otimes\Big[|V_{ub}|^2 (d+\overline{b}) + |V_{cb}|^2 (c+\overline{b})\Big]\\
                &+& 2\left(|V_{ub}|^2+|V_{cb}|^2\right)C_{2,g}\otimes g\Big\}\\
\\
F_{2,t}^{\overline\nu,n} &=& 2x\Big\{C_{2,q}\otimes\Big[|V_{td}|^2 (\overline{u} +t)+ |V_{ts}|^2(\overline{s}+t) + |V_{tb}|^2(\overline{b}+t)\Big]\\
                &+& 2\left(|V_{td}|^2 + |V_{ts}|^2 + |V_{tb}|^2\right)C_{2,g}\otimes g\Big\}
\end{array}
\end{equation}

Now, since the average of a given structure function
$F_2^{\nu(\overline{\nu})}$ is given by:
\begin{equation}
F_2^{\nu(\overline{\nu})} = fF_2^{\nu(\overline{\nu}),p}+(1-f)F_2^{\nu(\overline{\nu}),p}
\end{equation}
where:
\begin{equation}
f=\frac{N_p}{N_p+N_n}
\end{equation}
we have that:
\begin{equation}
\begin{array}{rcl}  
F_{2,l}^{\nu} &=& 2x\Big\{C_{2,q}\otimes\Big[|V_{ud}|^2 (fd+(1-f)u) +\left(|V_{ud}|^2+|V_{us}|^2\right)(f\overline{u}+(1-f)\overline{d}) + |V_{us}|^2 s\Big]\\
              &+& 2\left(|V_{ud}|^2+|V_{us}|^2\right)C_{2,g}\otimes g\Big\}\\
\\
F_{2,c}^{\nu} &=& 2x\Big\{C_{2,q}\otimes\Big[|V_{cd}|^2(fd+(1-f)u+\overline{c}) + |V_{cs}|^2 (s+\overline{c})\Big]\\
              &+& 2\left(|V_{cd}|^2+|V_{cs}|^2\right)C_{2,g}\otimes g\Big\}\\
\\
F_{2,b}^{\nu} &=& 2x\Big\{C_{2,q}\otimes\Big[|V_{ub}|^2 (f\overline{u}+(1-f)\overline{d}+b) + |V_{cb}|^2 (\overline{c}+b)\Big]\\
              &+& 2\left(|V_{ub}|^2+|V_{cb}|^2\right)C_{2,g}\otimes g\Big\}\\
\\
F_{2,t}^{\nu} &=& 2x\Big\{C_{2,q}\otimes\Big[|V_{td}|^2 (fd+(1-f)u +\overline{t})+ |V_{ts}|^2(s+\overline{t}) + |V_{tb}|^2(b+\overline{t})\Big]\\
              &+& 2\left(|V_{td}|^2 + |V_{ts}|^2 + |V_{tb}|^2\right)C_{2,g}\otimes g\Big\}
\end{array}
\end{equation}
and:
\begin{equation}
\begin{array}{rcl}  
F_{2,l}^{\overline\nu} &=& 2x\Big\{C_{2,q}\otimes\Big[|V_{ud}|^2 (f\overline{d}+(1-f)\overline{u}) +\left(|V_{ud}|^2+|V_{us}|^2\right)(fu+(1-f)d) + |V_{us}|^2 \overline{s}\Big]\\
              &+& 2\left(|V_{ud}|^2+|V_{us}|^2\right)C_{2,g}\otimes g\Big\}\\
\\
F_{2,c}^{\overline\nu} &=& 2x\Big\{C_{2,q}\otimes\Big[|V_{cd}|^2(f\overline{d}+(1-f)\overline{u}+c) + |V_{cs}|^2 (\overline{s}+c)\Big]\\
              &+& 2\left(|V_{cd}|^2+|V_{cs}|^2\right)C_{2,g}\otimes g\Big\}\\
\\
F_{2,b}^{\overline\nu} &=& 2x\Big\{C_{2,q}\otimes\Big[|V_{ub}|^2 (fu+(1-f)d+\overline{b}) + |V_{cb}|^2 (c+\overline{b})\Big]\\
              &+& 2\left(|V_{ub}|^2+|V_{cb}|^2\right)C_{2,g}\otimes g\Big\}\\
\\
F_{2,t}^{\overline\nu} &=& 2x\Big\{C_{2,q}\otimes\Big[|V_{td}|^2 (f\overline{d}+(1-f)\overline{u} +t)+ |V_{ts}|^2(\overline{s}+t) + |V_{tb}|^2(\overline{b}+t)\Big]\\
              &+& 2\left(|V_{td}|^2 + |V_{ts}|^2 + |V_{tb}|^2\right)C_{2,g}\otimes g\Big\}
\end{array}
\end{equation}


\subsection{Light Stucture Function}

At first, let's deal with the light structure function
$F_{2,l}^{\nu}$. We notice that it can be written as:
\begin{equation}
\begin{array}{rcl}  
F_{2,l}^{\nu} &=& 2x\Big\{C_{2,q}\otimes\Big[|V_{ud}|^2 \left(f(d-u+\overline{u}-\overline{d})+(u+\overline{d})\right) +|V_{us}|^2\left(f(\overline{u}-\overline{d})+(\overline{d}+s)\right)\Big]\\
              &+& \left(|V_{ud}|^2+|V_{us}|^2\right)C_{2,g}\otimes g\Big\}\\
\end{array}
\end{equation}

As usual we have to rotate the basis
$\{u,\overline{u},d,\overline{d},\dots\}$ into the basis
$\{\Sigma,g,V,T_3,V_3,\dots\}$, but knowing that:
\begin{equation}\label{ignmazio}
\begin{pmatrix}
u(\overline u)\\
d(\overline d)\\
s(\overline s)\\
c(\overline c)\\
b(\overline b)\\
t(\overline t)
\end{pmatrix}=\frac12\left[\begin{pmatrix}
u^+\\
d^+\\
s^+\\
c^+\\
b^+\\
t^+
\end{pmatrix}\pm\begin{pmatrix}
u^-\\
d^-\\
s^-\\
c^-\\
b^-\\
t^-
\end{pmatrix}\right]=
\frac{1}{120}
\begin{pmatrix}
 10 &  30 &  10 &   5 &   3 &   2 \\
 10 & -30 &  10 &   5 &   3 &   2 \\
 10 &   0 & -20 &   5 &   3 &   2 \\
 10 &   0 &   0 & -15 &   3 &   2 \\
 10 &   0 &   0 &   0 & -12 &   2 \\
 10 &   0 &   0 &   0 &   0 & -10
\end{pmatrix}
\begin{pmatrix}
\Sigma\pm V \\ T_3\pm V_3 \\ T_8\pm V_8 \\ T_{15}\pm V_{15} \\ T_{24}\pm V_{24} \\ T_{35}\pm V_{35}
\end{pmatrix}\,.
\end{equation}
we find that:
\begin{equation}
\begin{array}{l}
\displaystyle d - u = -\frac12(T_3+V_3)\\
\\
\displaystyle \overline{u}-\overline{d} = \frac12(T_3-V_3)
\end{array}
\end{equation}
so that:
\begin{equation}
d-u+\overline{u}-\overline{d} = - V_3
\end{equation}
while:
\begin{equation}
u+\overline{d} = \frac1{60}[10\Sigma+30V_3+10T_8+5T_{15}+3T_{24}+2T_{35}]
\end{equation}
and:
\begin{equation}
\overline{d}+s = \frac1{60}[10\Sigma-15(T_3-V_3)-5(T_8+3V_8)+5T_{15}+3T_{24}+2T_{35}]
\end{equation}
thus:
\begin{equation}
\begin{array}{rcl}  
F_{2,l}^{\nu} &=& \displaystyle 2x\Bigg\{C_{2,q}\otimes\Bigg[\frac1{60}|V_{ud}|^2 \left(10\Sigma+30(1-2f)V_3+10T_8+5T_{15}+3T_{24}+2T_{35}\right)\\
              &+& \displaystyle \frac1{60}|V_{us}|^2\left(10\Sigma-15(1-2f)(T_3-V_3)-5(T_8+3V_8)+5T_{15}+3T_{24}+2T_{35}\right)\Bigg]\\
              &+& \left(|V_{ud}|^2+|V_{us}|^2\right)C_{2,g}\otimes g\Bigg\}\\
\\
              &=& \displaystyle 2x\Bigg\{\frac1{60}C_{2,q}\otimes\Bigg[10K_l\Sigma-15(1-2f)\left[|V_{us}|^2T_3-(2|V_{ud}|^2+|V_{us}|^2)V_3\right]\\
              &+& \displaystyle 5(2|V_{ud}|^2-|V_{us}|^2)T_8-15|V_{us}|^2V_8+5K_lT_{15}+3K_lT_{24}+2K_lT_{35}\Bigg]\\
              &+& K_lC_{2,g}\otimes g\Bigg\}
\end{array}
\end{equation}
being:
\begin{equation}
K_l = 2\left(|V_{ud}|^2+|V_{us}|^2\right)
\end{equation}

Now let's consider $F_2^{\nu(\overline\nu)}$. First we notice that it
can be written as:
\begin{equation}
\begin{array}{rcl}  
F_{2,l}^{\overline\nu} &=& 2x\Big\{C_{2,q}\otimes\Big[|V_{ud}|^2 \left(f(\overline{d}-\overline{u}+u-d)+(\overline{u}+d)\right) +|V_{us}|^2\left(f(u-d)+(d+\overline{s})\right)\Big]\\
              &+& \left(|V_{ud}|^2+|V_{us}|^2\right)C_{2,g}\otimes g\Big\}\\
\end{array}
\end{equation}
but, from eq. (\ref{ignmazio}):
\begin{equation}
\begin{array}{l}
\displaystyle \overline{d} - \overline{u} = -\frac12(T_3-V_3)\\
\\
\displaystyle u-d = \frac12(T_3+V_3)
\end{array}
\end{equation}
so that:
\begin{equation}
\overline{d}-\overline{u}+u-d = V_3
\end{equation}
while:
\begin{equation}
\overline{u}+d = \frac1{60}[10\Sigma-30V_3+10T_8+5T_{15}+3T_{24}+2T_{35}]
\end{equation}
and:
\begin{equation}
d+\overline{s} = \frac1{60}[10\Sigma-15(T_3+V_3)-5(T_8-3V_8)+5T_{15}+3T_{24}+2T_{35}]
\end{equation}
thus:
\begin{equation}
\begin{array}{rcl}  
F_{2,l}^{\overline\nu} &=& \displaystyle 2x\Bigg\{\frac1{60}C_{2,q}\otimes\Bigg[10K_l\Sigma-15(1-2f)\left[|V_{us}|^2T_3+(2|V_{ud}|^2+|V_{us}|^2)V_3\right]\\
              &+& \displaystyle 5(2|V_{ud}|^2-|V_{us}|^2)T_8+15|V_{us}|^2V_8+5K_lT_{15}+3K_lT_{24}+2K_lT_{35}\Bigg]\\
              &+& K_lC_{2,g}\otimes g\Bigg\}
\end{array}
\end{equation}

\subsection{Charm Structure Function}

Now we consider the charm structure function for the neutrino. We see that it can be written as:
\begin{equation}
\begin{array}{rcl}
F_{2,c}^{\nu} &=& 2x\Big\{C_{2,q}\otimes\Big[|V_{cd}|^2(f(d-u)+u+\overline{c}) + |V_{cs}|^2 (s+\overline{c})\Big]\\
              &+& \left(|V_{cd}|^2+|V_{cs}|^2\right)C_{2,g}\otimes g\Big\}
\end{array}
\end{equation}
but:
\begin{equation}
\begin{array}{l}
\displaystyle d - u = -\frac12(T_3+V_3)
\end{array}
\end{equation}
while:
\begin{equation}
u+\overline{c} = \frac1{60}[10\Sigma+15(T_3+V_3)+5(T_8+V_8)-5(T_{15}-2V_{15})+3T_{24}+2T_{35}]
\end{equation}
and:
\begin{equation}
s+\overline{c} = \frac1{60}[10\Sigma-10(T_8+V_8)-5(T_{15}-2V_{15})+3T_{24}+2T_{35}]
\end{equation}
so that:
\begin{equation}
f(d - u)+u+\overline{c} = \frac1{60}[10\Sigma+15(1-2f)(T_3+V_3)+5(T_8+V_8)-5(T_{15}-2V_{15})+3T_{24}+2T_{35}]
\end{equation}
so we find that:
\begin{equation}
\begin{array}{rcl}  
F_{2,c}^{\nu} &=& \displaystyle 2x\Bigg\{\frac1{60}C_{2,q}\otimes\Bigg[10K_c\Sigma+15(1-2f)|V_{cd}|^2(T_3 + V_3)\\
              &+& \displaystyle  5(|V_{cd}|^2-2|V_{cs}|^2)(T_8+V_8) - 5K_c(T_{15}-2V_{15})+3K_cT_{24}+2K_cT_{35}\Bigg]\\
              &+& K_cC_{2,g}\otimes g\Bigg\}
\end{array}
\end{equation}
where we have defined:
\begin{equation}
K_c = |V_{cd}|^2+|V_{cs}|^2
\end{equation}

\subsection{Bottom Structure Function}

Now we consider the bottom structure function for the neutrino. We see
that it can be written as:
\begin{equation}
\begin{array}{rcl}
F_{2,b}^{\nu} &=& 2x\Big\{C_{2,q}\otimes\Big[|V_{ub}|^2 (f(\overline{u}-\overline{d})+\overline{d}+b) + |V_{cb}|^2 (\overline{c}+b)\Big]\\
              &+& \left(|V_{ub}|^2+|V_{cb}|^2\right)C_{2,g}\otimes g\Big\}
\end{array}
\end{equation}
but:
\begin{equation}
\begin{array}{l}
\displaystyle \overline{u} - \overline{d} = \frac12(T_3-V_3)
\end{array}
\end{equation}
while:
\begin{equation}
\overline{d}+b = \frac1{120}[20\Sigma-30(T_3-V_3)+10(T_8-V_8)+5(T_{15}-V_{15})-3(3T_{24}+5V_{24})+4T_{35}]
\end{equation}
and:
\begin{equation}
\overline{c}+b = \frac1{120}[20\Sigma-15(T_{15}-V_{15})-3(3T_{24}+5V_{24})+4T_{35}]
\end{equation}
so that:
\begin{equation}
f(\overline{u}-\overline{d})+\overline{d}+b = \frac1{120}[20\Sigma-30(1-2f)(T_3-V_3)+10(T_8-V_8)+5(T_{15}-V_{15})-3(3T_{24}+5V_{24})+4T_{35}]
\end{equation}
so we find that:
\begin{equation}
\begin{array}{rcl}  
F_{2,b}^{\nu} &=& \displaystyle 2x\Bigg\{\frac1{120}C_{2,q}\otimes\Bigg[20K_b\Sigma-30(1-2f)|V_{ub}|^2(T_3-V_3)+10|V_{ub}|^2(T_8-V_8)\\
              &+& \displaystyle 5(|V_{ub}|^2-3|V_{cb}|^2)(T_{15}-V_{15})-3K_b(3T_{24}+5V_{24})+4K_bT_{35}\Bigg]\\
              &+& K_cC_{2,g}\otimes g\Bigg\}
\end{array}
\end{equation}
where we have defined:
\begin{equation}
K_b = |V_{ub}|^2+|V_{cb}|^2
\end{equation}

\subsection{Top Structure Function}

Now we consider the top structure function for the neutrino. We see
that it can be written as:
\begin{equation}
\begin{array}{rcl}
F_{2,t}^{\nu} &=& 2x\Big\{C_{2,q}\otimes\Big[|V_{td}|^2 (f(d-u)+u +\overline{t})+ |V_{ts}|^2(s+\overline{t}) + |V_{tb}|^2(b+\overline{t})\Big]\\
              &+& \left(|V_{td}|^2 + |V_{ts}|^2 + |V_{tb}|^2\right)C_{2,g}\otimes g\Big\}
\end{array}
\end{equation}
but:
\begin{equation}
\begin{array}{l}
\displaystyle d-u = -\frac12(T_3+V_3)
\end{array}
\end{equation}
while:
\begin{equation}
u+\overline{t} = \frac1{120}[20\Sigma+30(T_3+V_3)+10(T_8+V_8)+5(T_{15}+V_{15})+3(T_{24}+V_{24})-4(2T_{35}-3V_{35})]
\end{equation}
and:
\begin{equation}
\begin{array}{l}
\displaystyle s+\overline{t} = \frac1{120}[20\Sigma-20(T_8+V_8)+5(T_{15}+V_{15})+3(T_{24}+V_{24})-4(2T_{35}-3V_{35})]\\
\\
\displaystyle b+\overline{t} = \frac1{120}[20\Sigma-12(T_{24}+V_{24})-4(2T_{35}-3V_{35})]
\end{array}
\end{equation}
so that:
\begin{equation}
f(d-u)+u +\overline{t} =\frac1{120}[20\Sigma+30(1-2f)(T_3+V_3)+10(T_8+V_8)+5(T_{15}+V_{15})+3(T_{24}+V_{24})-4(2T_{35}-3V_{35})]
\end{equation}
so we find that:
\begin{equation}
\begin{array}{rcl}  
F_{2,t}^{\nu} &=& \displaystyle 2x\Bigg\{\frac1{120}C_{2,q}\otimes\Bigg[20\Sigma+30(1-2f)|V_{td}|^2(T_3+V_3)+10(|V_{td}|^2-2|V_{ts}|)(T_8+V_8)\\
              &+& \displaystyle 5(|V_{td}|^2+|V_{ts}|^2)(T_{15}+V_{15})+3(|V_{td}|^2+|V_{ts}|^2-4|V_{tb}|^2)(T_{24}+V_{24})-4(2T_{35}-3V_{35})\Bigg]\\
              &+& K_cC_{2,g}\otimes g\Bigg\}
\end{array}
\end{equation}
where we have used the fact that:
\begin{equation}
|V_{td}|^2+|V_{ts}|^2+|V_{tb}|^2=1
\end{equation}

\section{$xF_3^{\nu(\overline{\nu})}$}

Now, we consider $xF_3^{\nu,p}$ which is defined as:

\begin{equation}\label{f3nup}
\begin{array}{rcl}
xF_3^{\nu,p} &=& 2x\Big\{ C_{3,q}\otimes\Big[\left(|V_{ud}|^2+|V_{cd}|^2+|V_{td}|^2\right)d\\
            &-& \left(|V_{ud}|^2+|V_{us}|^2+|V_{ub}|^2\right)\overline{u}\\
            &+& \left(|V_{us}|^2+|V_{cs}|^2+|V_{ts}|^2\right)s\\
            &-& \left(|V_{cd}|^2+|V_{cs}|^2+|V_{cb}|^2\right)\overline{c}\\
            &+& \left(|V_{ub}|^2+|V_{cb}|^2+|V_{tb}|^2\right)b\\
            &-& \left(|V_{td}|^2+|V_{ts}|^2+|V_{tb}|^2\right)\overline{t}\\
            &+& c^{CC}_g(N_f)C_{3,q}\otimes g\Big\}
\end{array}
\end{equation}
Note that, we are keeping the gluon term which in the ZM-VFNS would be zero. On the other hand in the massive scheme it's not the case.

Again, we want to split up this structure function into its light part
plus the heavy parts, that is:
\begin{equation}
xF_3^{\nu,p} = xF_{3,l}^{\nu,p} + xF_{3,c}^{\nu,p} + xF_{3,b}^{\nu,p} + xF_{3,t}^{\nu,p}
\end{equation}

Using the same tricks of the previous Section, we get:
\begin{equation}
\begin{array}{rcl}  
xF_{3,l}^{\nu,p} &=& 2x\Big\{C_{3,q}\otimes\Big[|V_{ud}|^2 d -\left(|V_{ud}|^2+|V_{us}|^2\right)\overline{u} + |V_{us}|^2 s\Big]\\
                 &+& \left(|V_{ud}|^2+|V_{us}|^2\right)C_{3,g}\otimes g\Big\}\\
\\
xF_{3,c}^{\nu,p} &=& 2x\Big\{C_{3,q}\otimes\Big[|V_{cd}|^2(d-\overline{c}) + |V_{cs}|^2 (s-\overline{c})\Big]\\
                 &+& \left(|V_{cd}|^2+|V_{cs}|^2\right)C_{3,g}\otimes g\Big\}\\
\\
xF_{3,b}^{\nu,p} &=& 2x\Big\{C_{3,q}\otimes\Big[|V_{ub}|^2(-\overline{u}+b) + |V_{cb}|^2 (-\overline{c}+b)\Big]\\
                 &+& \left(|V_{ub}|^2+|V_{cb}|^2\right)C_{3,g}\otimes g\Big\}\\
\\
xF_{3,t}^{\nu,p} &=& 2x\Big\{C_{3,q}\otimes\Big[|V_{td}|^2 (d -\overline{t})+ |V_{ts}|^2(s-\overline{t}) + |V_{tb}|^2(b-\overline{t})\Big]\\
                 &+& \left(|V_{td}|^2 + |V_{ts}|^2 + |V_{tb}|^2\right)C_{3,g}\otimes g\Big\}
\end{array}
\end{equation}
and:
\begin{equation}
\begin{array}{rcl}  
xF_{3,l}^{\overline\nu,p} &=& 2x\Big\{C_{3,q}\otimes\Big[-|V_{ud}|^2 \overline{d} +\left(|V_{ud}|^2+|V_{us}|^2\right)u - |V_{us}|^2 \overline{s}\Big]\\
                 &+& \left(|V_{ud}|^2+|V_{us}|^2\right)C_{3,g}\otimes g\Big\}\\
\\
xF_{3,c}^{\overline\nu,p} &=& 2x\Big\{C_{3,q}\otimes\Big[|V_{cd}|^2(-\overline{d}+c) + |V_{cs}|^2 (-\overline{s}+c)\Big]\\
                 &+& \left(|V_{cd}|^2+|V_{cs}|^2\right)C_{3,g}\otimes g\Big\}\\
\\
xF_{3,b}^{\overline\nu,p} &=& 2x\Big\{C_{3,q}\otimes\Big[|V_{ub}|^2(u-\overline{b}) + |V_{cb}|^2 (c-\overline{b})\Big]\\
                 &+& \left(|V_{ub}|^2+|V_{cb}|^2\right)C_{3,g}\otimes g\Big\}\\
\\
xF_{3,t}^{\overline\nu,p} &=& 2x\Big\{C_{3,q}\otimes\Big[|V_{td}|^2 (-\overline{d}+t)+ |V_{ts}|^2(-\overline{s}+t) + |V_{tb}|^2(\overline{b}+t)\Big]\\
                 &+& \left(|V_{td}|^2 + |V_{ts}|^2 + |V_{tb}|^2\right)C_{3,g}\otimes g\Big\}
\end{array}
\end{equation}

While for the neutron we have:
\begin{equation}
\begin{array}{rcl}  
xF_{3,l}^{\nu,n} &=& 2x\Big\{C_{3,q}\otimes\Big[|V_{ud}|^2 u -\left(|V_{ud}|^2+|V_{us}|^2\right)\overline{d} + |V_{us}|^2 s\Big]\\
                 &+& \left(|V_{ud}|^2+|V_{us}|^2\right)C_{3,g}\otimes g\Big\}\\
\\
xF_{3,c}^{\nu,n} &=& 2x\Big\{C_{3,q}\otimes\Big[|V_{cd}|^2(u-\overline{c}) + |V_{cs}|^2 (s-\overline{c})\Big]\\
                 &+& \left(|V_{cd}|^2+|V_{cs}|^2\right)C_{3,g}\otimes g\Big\}\\
\\
xF_{3,b}^{\nu,n} &=& 2x\Big\{C_{3,q}\otimes\Big[|V_{ub}|^2(-\overline{d}+b) + |V_{cb}|^2 (-\overline{c}+b)\Big]\\
                 &+& \left(|V_{ub}|^2+|V_{cb}|^2\right)C_{3,g}\otimes g\Big\}\\
\\
xF_{3,t}^{\nu,n} &=& 2x\Big\{C_{3,q}\otimes\Big[|V_{td}|^2 (u -\overline{t})+ |V_{ts}|^2(s-\overline{t}) + |V_{tb}|^2(b-\overline{t})\Big]\\
                 &+& \left(|V_{td}|^2 + |V_{ts}|^2 + |V_{tb}|^2\right)C_{3,g}\otimes g\Big\}
\end{array}
\end{equation}
and:
\begin{equation}
\begin{array}{rcl}  
xF_{3,l}^{\overline\nu,n} &=& 2x\Big\{C_{3,q}\otimes\Big[-|V_{ud}|^2 \overline{u} +\left(|V_{ud}|^2+|V_{us}|^2\right)d - |V_{us}|^2 \overline{s}\Big]\\
                 &+& \left(|V_{ud}|^2+|V_{us}|^2\right)C_{3,g}\otimes g\Big\}\\
\\
xF_{3,c}^{\overline\nu,n} &=& 2x\Big\{C_{3,q}\otimes\Big[|V_{cd}|^2(-\overline{u}+c) + |V_{cs}|^2 (-\overline{s}+c)\Big]\\
                 &+& \left(|V_{cd}|^2+|V_{cs}|^2\right)C_{3,g}\otimes g\Big\}\\
\\
xF_{3,b}^{\overline\nu,n} &=& 2x\Big\{C_{3,q}\otimes\Big[|V_{ub}|^2(d-\overline{b}) + |V_{cb}|^2 (c-\overline{b})\Big]\\
                 &+& \left(|V_{ub}|^2+|V_{cb}|^2\right)C_{3,g}\otimes g\Big\}\Big\}\\
\\
xF_{3,t}^{\overline\nu,n} &=& 2x\Big\{C_{3,q}\otimes\Big[|V_{td}|^2 (-\overline{u}+t)+ |V_{ts}|^2(-\overline{s}+t) + |V_{tb}|^2(\overline{b}+t)\Big]\\
                 &+& \left(|V_{td}|^2 + |V_{ts}|^2 + |V_{tb}|^2\right)C_{3,g}\otimes g\Big\}
\end{array}
\end{equation}

So, combining proton and neutron structure functions we get:
\begin{equation}
\begin{array}{rcl}  
xF_{3,l}^{\nu} &=& 2x\Big\{C_{3,q}\otimes\Big[|V_{ud}|^2 (fd+(1-f)u) -\left(|V_{ud}|^2+|V_{us}|^2\right)(f\overline{u}+(1-f)\overline{d}) + |V_{us}|^2 s\Big]\\
                 &+& \left(|V_{ud}|^2+|V_{us}|^2\right)C_{3,g}\otimes g\Big\}\\
\\
xF_{3,c}^{\nu} &=& 2x\Big\{C_{3,q}\otimes\Big[|V_{cd}|^2(fd+(1-f)u-\overline{c}) + |V_{cs}|^2 (s-\overline{c})\Big]\\
                 &+& \left(|V_{cd}|^2+|V_{cs}|^2\right)C_{3,g}\otimes g\Big\}\\
\\
xF_{3,b}^{\nu} &=& 2x\Big\{C_{3,q}\otimes\Big[|V_{ub}|^2(-f\overline{u}-(1-f)\overline{d}+b) + |V_{cb}|^2 (-\overline{c}+b)\Big]\\
                 &+& \left(|V_{ub}|^2+|V_{cb}|^2\right)C_{3,g}\otimes g\Big\}\Big\}\\
\\
xF_{3,t}^{\nu} &=& 2x\Big\{C_{3,q}\otimes\Big[|V_{td}|^2 (fd +(1-f)u -\overline{t})+ |V_{ts}|^2(s-\overline{t}) + |V_{tb}|^2(b-\overline{t})\Big]\\
                 &+& \left(|V_{td}|^2 + |V_{ts}|^2 + |V_{tb}|^2\right)C_{3,g}\otimes g\Big\}
\end{array}
\end{equation}
and:
\begin{equation}
\begin{array}{rcl}  
xF_{3,l}^{\overline\nu} &=& 2x\Big\{C_{3,q}\otimes\Big[-|V_{ud}|^2 (f\overline{d}+(1-f)\overline{u}) +\left(|V_{ud}|^2+|V_{us}|^2\right)(fu+(1-f)d) - |V_{us}|^2 \overline{s}\Big]\\
                 &+& \left(|V_{ud}|^2+|V_{us}|^2\right)C_{3,g}\otimes g\Big\}\\
\\
xF_{3,c}^{\overline\nu} &=& 2x\Big\{C_{3,q}\otimes\Big[|V_{cd}|^2(-f\overline{d}-(1-f)\overline{u}+c) + |V_{cs}|^2 (-\overline{s}+c)\Big]\\
                 &+& \left(|V_{cd}|^2+|V_{cs}|^2\right)C_{3,g}\otimes g\Big\}\\
\\
xF_{3,b}^{\overline\nu} &=& 2x\Big\{C_{3,q}\otimes\Big[|V_{ub}|^2(fu+(1-f)d-\overline{b}) + |V_{cb}|^2 (c-\overline{b})\Big]\\
                 &+& \left(|V_{ub}|^2+|V_{cb}|^2\right)C_{3,g}\otimes g\Big\}\Big\}\\
\\
xF_{3,t}^{\overline\nu} &=& 2x\Big\{C_{3,q}\otimes\Big[|V_{td}|^2 (-f\overline{d}-(1-f)\overline{u}+t)+ |V_{ts}|^2(-\overline{s}+t) + |V_{tb}|^2(\overline{b}+t)\Big]\\
                 &+& \left(|V_{td}|^2 + |V_{ts}|^2 + |V_{tb}|^2\right)C_{3,g}\otimes g\Big\}
\end{array}
\end{equation}

At this point we notice that $xF_3^{\nu(\overline\nu)}$, a part from
the fact that there are $C_{3,q}$ and $C_{3,g}$ rather than $C_{2,q}$
and $C_{2,g}$, is almost equal to $F_2^{\nu(\overline\nu)}$ but with
every antiflavour having opposite sign. Now, starting from
eq. (\ref{ignmazio}), we can write that:
\begin{equation}\label{ignmazio1}
\begin{pmatrix}
u(-\overline u)\\
d(-\overline d)\\
s(-\overline s)\\
c(-\overline c)\\
b(-\overline b)\\
t(-\overline t)
\end{pmatrix}=
\frac{1}{120}
\begin{pmatrix}
 10 &  30 &  10 &   5 &   3 &   2 \\
 10 & -30 &  10 &   5 &   3 &   2 \\
 10 &   0 & -20 &   5 &   3 &   2 \\
 10 &   0 &   0 & -15 &   3 &   2 \\
 10 &   0 &   0 &   0 & -12 &   2 \\
 10 &   0 &   0 &   0 &   0 & -10
\end{pmatrix}
\begin{pmatrix}
V\pm \Sigma \\ V_3\pm T_3 \\ V_8\pm T_8 \\ V_{15}\pm T_{15} \\ V_{24}\pm T_{24} \\ V_{35}\pm T_{35}
\end{pmatrix}\,.
\end{equation}
So, we see that replacing $\overline{q}$ with $-\overline{q}$ in the
basis $\{u,\overline{u},d,\overline{d},\dots\}$ is equivalent to
exchange $\Sigma\leftrightarrow V$, $T_3\leftrightarrow V_3$,
$T_8\leftrightarrow V_8$ and so on in the basis
$\{\Sigma,g,V,T_3,V_3,\dots\}$. For this reason we can directly write
down the expressions for $xF_3^{\nu(\overline{\nu})}$.

\section{Heavy Quark Structure Functions in the Massive Scheme}

In this sectio we will try to understand how the structure of the
heavy quark structure functions change in the massive scheme where the
heavy quark PDFs are absent. In principle, this step, if computing
structure functions in the purely massive scheme, is not needed
becuase the PDF evolution would automatically adjust the structure of
the observables. However, when considering the FONLL scheme where
massive and zero-mass schemes are combined and PDFs evolve in the
ZM-VFNS, it is necessary to know how massive structure functions are
written in terms of PDFs in the evolution basis to ensure a proper
combination.

It is possible to write a set of simple rules that allow us to write
the massive structure functions starting from the expressions we found
in the previous sections. The principle is simple: in the $N_f$
massive scheme, all PDFs from $N_f+1$ to 6 are absent. In order to see
what happens at the level of PDFs in the evolution basis, we need to
consider all the cases from $N_f=3$ to $N_f=5$ ($N_f=6$ is equivalent
to the massless scheme).

In the $N_f=3$ massive scheme we have:
\begin{equation}
\begin{array}{l}
T_{15}\rightarrow \Sigma\\
T_{24}\rightarrow \Sigma\\
T_{35}\rightarrow \Sigma\\
V_{15}\rightarrow V\\
V_{24}\rightarrow V\\
V_{35}\rightarrow V
\end{array}
\end{equation}

In the $N_f=4$ massive scheme:
\begin{equation}
\begin{array}{l}
T_{24}\rightarrow \Sigma\\
T_{35}\rightarrow \Sigma\\
V_{24}\rightarrow V\\
V_{35}\rightarrow V
\end{array}
\end{equation}

Finally, in the $N_f=5$ massive scheme:
\begin{equation}
\begin{array}{l}
T_{35}\rightarrow \Sigma\\
V_{35}\rightarrow V
\end{array}
\end{equation}

\end{document}
